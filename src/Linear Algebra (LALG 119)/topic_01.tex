\topic{Introduction to set theory}{Introduction to set theory}

\subsection{Definitions for set and subset}
\begin{definition}
    A set is a collection of objects about which is possible to determine weather or not a particular object is a member of the set.
\end{definition}

Sets are usually denoted by capital letters, and the objects within them are referred to as \textbf{elements}. One way to describe a set is by giving a list of their elements, just as the example below.
\begin{example}[Examples of sets are]
    \begin{equation}
        A = \lbrace 1, 2, 3, 4\rbrace \quad\quad\quad B = \lbrace 1, 2, 3, 4\ldots\rbrace \quad\quad\quad C = \lbrace 0, 2, 4, 6\ldots\rbrace
    \end{equation}
\end{example}

Another way to describe a set is by the so called \textbf{set-builder notation}. This notation uses braces to enclose a property that is the qualification for membership in the set.

\begin{example}[Sets described by the set-builder notation.]
    \begin{align}
        A &= \lbrace x\tq x \textrm{ is a natural number } \rbrace = \lbrace 0, 1, 2\ldots\rbrace \\
        B &= \lbrace x\tq x \textrm{ is an even natural number } \rbrace = \lbrace x : 2\tq x \textrm{ and $x$ is natural } \rbrace
    \end{align}
\end{example}

\begin{note}
    In this previous example, both symbols $\tq$ and $ : $ are used to denote \textit{such as}.
\end{note}

We write $a\in A$ if $a$ is an element in the set $A$. Otherwise, we write $a\notin A$ to mean that $a$ is not an element in the set $A$. For instance, 2/3 is a rational number, but not an integer one. Then
\begin{equation}
    \frac{2}{3}\in\Q,\quad\textrm{ but } \frac{2}{3}\notin\Z 
\end{equation}

We use $\O$ to refer to the set with no elements, denominated the \textbf{empty set}.

\begin{definition}
    Let $A$ and $B$ be sets. We say that $A$ is a subset of $B$ if every element in $A$ is an element in $B$. If $A$ is a subset of $B$ we write $A\subset B$ and we say \textit{$A$ is contained in $B$} or \textit{$B$ contains $A$}. Otherwise we write $A\notin B$.
\end{definition}
\begin{example}[Some examples of subsets.]
    \begin{align}
        &\N\subset\Z\subset\Q\subset\R\subset\C \\
        &\O\in C\textrm{, where $C$ is any set.} \\
        &A = \{1, 2, 3\} \subset B = \{1, 2, 3, 4, 5\} 
    \end{align}
\end{example}

\begin{definition}
    Two sets $A$ and $B$ are equal if $A \subset B$ and $B\subset A$; i.e. if they have the same elements.
\end{definition}
\begin{definition}
    We say that $A$ is properly contained in $B$ if $A\subset B$ but $A\neq B$. For example, $\N$ is properly contained in $\Z$ because $\N\subset\Z$, but $\N\neq\Z$.
\end{definition}
\begin{note}
    We use $:=$ to mean \textit{by definition}. For example, $\N := \{0, 1, 2\ldots\} $.
\end{note}

\subsection{The power set of a set}
\begin{definition}
    Let $A$ be a set. The power set of $A$ is a set whose elements are all the subsets of $A$, and we denote it by $P\left( A \right) $
\end{definition}
\begin{example}[Power set of $A = \{a, b, c\} $.]
    \begin{equation}
    P(A) = \lbrace \O, \{a\}, \{b\}, \{c\}, \{a, b\}, \{a, c\}, \{b, c\}, \{a, b, c\}\rbrace
    \end{equation}
\end{example}

In this previous example, we must keep notice that $\O\in P\left( A \right)$ and $\O\subset P\left( A \right)$ are not exactly the same thing. In the first one we are refering to the $\O$ element in the set $P\left( A \right)$; while in the second, $\O$ is the empty set, which as seen previously is a subset of any set. $\{\O\}\subset P\left( A \right)$ has also a different meaning, as $\O$ in this case is the element contained in $P\left( A \right)$. We couldn't write $\{\O\}\in P\left( A \right)$ because the element $\{\O\}$ is not contained in $P\left( A \right) $.

\subsection{Unions and intersections of sets}
\begin{definition}
    \textbf{(Union of sets).} Let $A$ and $B$ be two sets. The union of $A$ and $B$ is another set whose elements are the elements in $A$ and the elements in $B$. We denote this set by $A\cup B$.
\end{definition}
\begin{example}[Let $A = \{a, b, c\} $ and $B = \{b, h, f\} $]
    \begin{equation}
        A\cup B = \{a, b, c, h, f\} 
    \end{equation}
\end{example}

Another way to define the union of two sets is A\cup B := \{x\tq x\in A\textrm{ or } x\in B\}.

\begin{definition}
    \textbf{(Intersection of sets).} Let $A$ and $B$ be two sets. The intersection of $A$ and $B$ is another set whose elements are the elements that $A$ and $B$ have in common. We denote this set by $A\cap B$. 
\end{definition}

We can also define the intersection of two sets as $A\cap B := \{x\tq x\in A\textrm{ and } x\in B\} $.

\begin{example}[Let $A\cap B = \{b\} $ and $C = \{x, y, z\} $.]
    \begin{equation}
        A\cap C = \O \quad\rightarrow\quad\textrm{ we say that $A$ and $C$ are disjoint.}
    \end{equation}
\end{example}

\subsubsection{Some properties of unions and intersections}
Let $A$, $B$ and $C$ be three non empty sets. The following properties are hold related to the union and intersection operations between those sets.
\newpage
\bgroup
\def\arraystretch{1.5}
\def\tabcolsep{20}
\begin{table}[h!]
\centering
\begin{tabular}{|c|c|}
    \hline
    \textbf{Property name} & \textbf{Properties} \\
    \hline
    Commutativity & $A\cap B = B\cap A$ \\
    \hline
    Commutativity & $A\cup B = B\cup A$ \\
    \hline
    Idempotency & $A\cup A = A$ \\
    \hline
    Idempotency & $A\cap A = A$ \\
    \hline
    Disjoint sets & If $A\cap B = \O$ we say they are disjoint. \\
    \hline
    Associativity & $\left( A\cap B \right)\cap C = A\cap\left( B\cap C \right)$ \\
    \hline
    Associativity & $\left( A\cup B \right)\cup C = A\cup\left( B\cup C \right)$ \\
    \hline
    Distributivity & $A\cup\left( B\cap C\right) = \left( A\cup B \right)\cap\left( A\cup C \right)$ \\
    \hline
    Distributivity & $A\cap\left( B\cup C\right) = \left( A\cap B \right)\cup\left( A\cap C \right)$ \\
    \hline
    Cancelation & $A\cup\left( B\cap A \right) = A$ \\
    \hline
    Cancelation & $A\cap\left( B\cup A \right) = A$ \\
    \hline
\end{tabular}
\caption{Some properties of unions and intersections of sets.}
\end{table}
\egroup

\hide{
\subsubsection{Some properties of unions and intersections}
\begin{prop}
    Commutativity. $\quad A\cap B = B\cap A, \quad A\cup B = B\cup A$.
\end{prop}
\begin{prop}
    Idempotency. $\quad A\cup A = A, \quad A\cap A = A$.
\end{prop}
\begin{prop}
    If $A\cap B = \O$ we say they are disjoint.
\end{prop}
\begin{prop}
    Associativity. 
    \begin{equation*}
        \quad \left( A\cap B \right)\cap C = A\cap \left( B\cap C \right) \quad\quad\quad \left( A\cup B \right)\cup C = A\cup \left( B\cup C \right)
    \end{equation*}
\end{prop}
\begin{prop}
    Distributivity.
    \begin{equation}
        A\cup \left( B\cap C \right) = \left( A\cup B \right)\cap\left( A\cup C \right) \quad\quad\quad A\cap \left( B\cup C \right) = \left( A\cap B \right)\cup\left( A\cap C \right)   
    \end{equation}
\end{prop}
\begin{prop}
    Cancelation. $\quad A\cup \left( B\cap A \right) = A, \quad A\cap\left( B\cup A \right) = A $.
\end{prop}
}

\subsection{Venn diagrams and universal sets}
Sometimes it is convenient to draw a picture of the sets we are working with. When we do this we are assuming that the sets that we are considering are subsets of some universal set which we denote by $\mathcal{U}$.

\begin{example}[Suppose we are considering sets $A$ and $B$.] 
    Then, we could draw the following picture, denominated \textbf{Venn diagram}.
\end{example}
\begin{remark}
    For any set $A,\quad A\cap\mathcal{U} = A,\quad\mathcal{U}\cup A = \uset$.
\end{remark}

\begin{definition}
    The complement of a set $A$ is $A^C:=\{a\in\uset\tq a\notin A\}$.
\end{definition}
\begin{example}[Let $\uset = \Z$.]
    \begin{equation}
        \N\subset\uset, \quad\quad\N^C = \{x\in\Z\tq x<0\}
    \end{equation}
\end{example}

\begin{definition}
\textbf{(De Morgan's laws)}. Let $A$ and $B$ be two sets in $\uset$. Then, the following equalities are hold.
\begin{equation}\label{demorgan:1}
    \left( A\cup B \right)^C = A^C\cap B^C
\end{equation}
\begin{equation}\label{demorgan:2}
    \left( A\cap B \right)^C = A^C\cup B^C 
\end{equation}
\end{definition}
\begin{proof}
    \textit{(First De Morgan's law)}. In order to proof the first De Morgan's law we need first to proof that $\left( A\cup B \right)^C\subset A^C\cap B^C \textrm{ and then } A^C\cap B^C \subset \left(  A\cup B\right)^C$. Only then we can say $\left( A\cup B \right)^C = A^C\cap B^C $. 
    %Let $a\in\left( A\cup B \right)^C,\ a\in A^C\cap B^C$ ? So we take
    \begin{align}
        \textrm{Let } a\in\left( A\cup B \right)^C &\implies a\in\uset\textrm{ and } a\notin A\cup B \implies \\
                                     &\implies a\in\uset\textrm{ and } a\notin A\textrm{ and } a\notin B \implies \\ &\implies a\in A^C\textrm{ and } a\in B^C \implies \\
                                                         &\implies a\in A^C\cap B^C
    \end{align}
    So, we can afirm that $\left( A\cup B \right)^C \subset A^C\cap B^C $.
    \begin{align}
        \textrm{Now, let } b\in A^C\cap B^C &\implies b\in A^C\textrm{ and } b\in B^C \implies \\
                                            &\implies b\in\uset\textrm{ but } a\notin A\textrm{ and } a\notin B \implies \\ &\implies b\in\uset\textrm{ and } b\notin A\cup B \implies \\
                                            &\implies b\in \left( A\cup B \right)^C
    \end{align}
    So, we can afirm that $A^C\cap B^C\subset\left( A\cup B \right)^C $. Therefore,
    \begin{equation}
        \left( A\cup B \right)^C = A^C\cap B^C
    \end{equation}

\end{proof}

\begin{proof}
    \textit{(Second De Morgan's law)}. In order to proof the second De Morgan's law we need first to proof that $\left( A\cap B \right)^C\subset A^C\cup B^C \textrm{ and then } A^C\cup B^C \subset \left(  A\cap B\right)^C$. Only then we can say $\left( A\cap B \right)^C = A^C\cup B^C $. 
    %Let $a\in\left( A\cup B \right)^C,\ a\in A^C\cap B^C$ ? So we take
    \begin{align}
        \textrm{Let } a\in\left( A\cap B \right)^C &\implies a\in\uset\textrm{ and } a\notin A\cap B \implies \\
                                     &\implies a\in\uset\textrm{ and } a\notin A\textrm{ or } a\notin B \implies \\ &\implies a\in A^C\textrm{ or } a\in B^C \implies \\
                                                         &\implies a\in A^C\cup B^C
    \end{align}
    So, we can afirm that $\left( A\cap B \right)^C \subset A^C\cup B^C $.
    \begin{align}
        \textrm{Now, let } b\in A^C\cup B^C &\implies b\in A^C\textrm{ or } b\in B^C \implies \\
                                            &\implies b\in\uset\textrm{ but } a\notin A\textrm{ or } a\notin B \implies \\ &\implies b\in\uset\textrm{ and } b\notin A\cap B \implies \\
                                            &\implies b\in \left( A\cap B \right)^C
    \end{align}
    So, we can afirm that $A^C\cup B^C\subset\left( A\cap B \right)^C $. Therefore,
    \begin{equation}
        \left( A\cap B \right)^C = A^C\cup B^C
    \end{equation}

\end{proof}

\subsection{Partitions of sets}
\begin{definition}
    Let $A$ be a non-empty set. A partition of $A$ is a separation of $A$ into mutually disjoint non-empty subsets.
\end{definition}
\begin{example}[Let $A = \{a, b, c, d, e\} $.]
    A partition of $A$ could be
    \begin{equation}
        B_1 = \{a\}, \quad B_2 = \{b, e, d\}, \quad\ B_3 = \{c\}, \quad\quad B_1\cup B_2\cup B_3 = A
    \end{equation}
    \begin{equation}
        B_i\cap B_j = \O\quad\textrm{for}\quad i\neq j,\quad\quad B_i\neq\O\quad\textrm{for}\quad i = 1, 2, 3
    \end{equation}
\end{example}

\subsection{Other operations with sets}
\begin{definition}
    \textbf{(Difference of sets).} Let $A$ and $B$ be two sets. The difference of $A$ and $B$ is another set, $A\textbackslash B$, whose elements are the elements in $A$ which are not contained in $B$. 
    \begin{equation}
        A\textbackslash B := \{a\in A \tq a\notin B \} 
    \end{equation}
\end{definition}
\begin{definition}
    \textbf{(Symmetric difference of sets).} Let $A$ and $B$ be two sets. The symmetric difference of $A$ and $B$ is another set, $A\triangle B$, whose elements in $A$ that are not contained in $B$ and the elements in $B$ that are not contained in $A$.
    \begin{equation}
        A\triangle B := \{a\in A\tq a\notin B \}\cup \{b\in B\tq b\notin A \}  
    \end{equation}
\end{definition}
\begin{remark}
    $A\triangle B = A\cup B \textbackslash A\cap B$.
\end{remark}
\begin{example}[Let $A = \{1, 2, 3, 4\} $ and $B = \{3, 5, 7\} $.]
    \begin{equation}
        A\textbackslash B = \{1, 2, 4\} 
    \end{equation}
    \begin{equation}
        A\triangle B = \{1, 2, 4\}\cup \{5, 7\} = \{1, 2, 4, 5, 7\}   
    \end{equation}
\end{example}

\begin{proposition}
    $B\triangle A = \left( A\cap B \right)^C $.
\end{proposition}
\begin{proof}
    Let $A = \{a, b, c\} $ and $B = \{c, d\} $. Consider $\uset = A\cup B$.
    \begin{align}
        B\triangle A &= \{d\}\cup \{a, b\} = \{a, b, d\} \\
        A\cap B &= \{c\} \quad\rightarrow\quad \left( A\cap B \right)^C = \{a, b, d\} 
    \end{align}
    Therefore, the equality $B\triangle A = \left( A\cap B \right)^C $ is hold considering $\uset = A\cup B$.
\end{proof}

\begin{definition}
    \textbf{(Cartesian product of sets).} Let $A$ and $B$ be two sets. The cartesian product of $A$ and $B$ is the set of the ordered pairs of the form $\left( a, b \right)$ where $a\in A$ and $b\in B$.
    \begin{equation}
        A\times B := \{\left( a, b \right) \tq a\in A,\ b\in B\} 
    \end{equation}
\end{definition}

\begin{example}[Let $A = \{a, b, c\} $ and $B = \{c, 3\} $.]
    \begin{equation}
        A\times B = \{\left( a, c \right), \left( a, 3 \right), \left( b, c \right), \left( b, 3 \right), \left( c, c \right), \left( c, 3 \right)\}
    \end{equation}
\end{example}
\begin{note}
    In general, $B\times A\neq A\times B$.
\end{note}

\subsection{Cardinality of a set}
\begin{definition}
    Let $A$ and $B$ be two sets. The cardinality of $A$, $\textrm{card}\left( A \right) $, is the number of elements in $A$.
    \begin{equation}
        \textrm{If card}\left( A \right) < \infty\textrm{ and } \textrm{card}\left( B \right) < \infty \implies \textrm{card}\left( A\times B \right) = \textrm{ card}\left( A \right)\cdot \textrm{ card}\left( B \right)   
    \end{equation}
\end{definition}
