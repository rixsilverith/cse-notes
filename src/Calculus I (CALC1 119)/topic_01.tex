\topic{Definición axiomática del \\ conjunto \R}{Definición axiomática del conjunto \R} 
\subsection{Números naturales, enteros y racionales. Compatibilidad con las operaciones básicas}
Empezando por los números naturales, con la necesidad de hacer operaciones, y sus respectivas inversas, a lo largo de la historia se han ido construyendo conjuntos de números cada vez más grandes, pasando por los enteros, $\Z$, y los racionales,
\begin{equation}
    \Q &:= \lbrace \frac{m}{n}\tq m, n\in\Z, n\neq 0 \rbrace
\end{equation}
\begin{observacion}
    En algunos libros $\N$ no contiene el cero. En esta asignatura siempre incluiremos el cero en $\N$, de modo que indicaremos con $\N\textbackslash \{0\}:=\N^* $ el conjunto de los naturales, excluido el cero.
\end{observacion}

Que un conjunto de números \textit{esté cerrado} para cierta operación significa que al hacer dicha operación entre dos elementos del conjunto, el elemento resultante pertenece también al mismo conjunto. Dicho esto, analizamos para que operaciones están cerrados los distintos conjuntos de números.
\begin{itemize}
    \item Con los naturales, $\N$, es posible sumar, pero no restar.
        \begin{equation}
            3-2 = 1\in\N\quad\quad 2 - 3 = -1 \notin\N 
        \end{equation}
    \item Con los enteros, $\Z$, es posible sumar y restar. Además, es posible multiplicar, pero no dividir.
        \begin{equation}
            3\cdot 2 = 6\in\Z\quad\quad \left( -3 \right)\cdot 2 = -6\in\Z \quad\quad \frac{3}{4}\notin\Z 
        \end{equation}
    \item Finalmente, con los racionales, $\Q$, podemos realizar las cuatro operaciones básicas, podemos sumar, restar, multiplicar y dividir. 
\end{itemize}

Puesto que el conjunto $\Q$ está cerrado para las cuatro operaciones básicas, surge la pregunta: ¿Por qué necesitamos also más que los racionales? ¿Existen realmente números que no son fracciones o que no se pueden obtener a partir de ellas mediante las cuatro operaciones?

\subsection{Insuficiencia de los números racionales}
Para dar una explicación satisfactoria de los principales conceptos del análisis matemático como son la convergencia, la continuidad, la diferenciabilidad y la integrabilidad, es necesario que estos conceptos esten basados en un conjunto de números precisamente definido.

El conjunto de los números racionales es inadecuado para muchos propósitos, como campo y como conjunto ordenado. Por ejemplo, no existe un número racional $\alpha$ tal que $\alpha^2 = 2$. Esto lleva a la introducción de los llamados \textit{números irracionales}, que suelen ser escritos como una sucesión "aproximada" de decimales. Así, la sucesión
\begin{equation}
    1,\ 1'4,\ 1'41,\ 1'414,\ 1'4142,\ \ldots
\end{equation}
podemos decir que "tiende a $\sqrt{2}$". Sin embargo, a no ser que el número irracional $\sqrt{2}$ haya sido claramente definido, debe surgirnos una pregunta: ¿Qué leches es a lo que "tiende" esta sucesión? Esta pregunta puede ser respondida tan pronto como el \textit{sistema de números reales}, \R,  sea construido.

Consideremos un problema muy sencillo: calcular la longitud de la diagonal de un cuadrado de la lado $1$. Sabemos gracias al teorema de Pitágoras que $a^2 + b^2 = c^2$, en nuestro caso $1^2 + 1^2 = 2$. Por lo tanto, la longitud $c$ de la hipotenusa es tal que $c^2 = 2$, es decir, $c = \sqrt{2}$. Que no existe en $\Q$ ningún número que verifique dicha igualdad es lo que vamos a demostrar a continuación, y que $\sqrt{2}\notin\Q$, pero antes definimos el siguiente lema.
\begin{lema}
    Si $p^2$ es par $\quad\implies\quad p$ es par. 
\end{lema}
\begin{proof}
    Probamos por contradicción que si $p$ es impar, entonces $p^2$ también resulta ser impar. Si $p$ es un número impar entonces podemos definirlo como 
    \begin{equation}
        \label{p_def}
        p = 2n + 1, \quad n\in\Z
    \end{equation}
    entonces, si $p^2$ es también impar debería tener la forma 
    \begin{equation}
        p^2 = 2m + 1, \quad m\in\Z
    \end{equation}
    Elevando al cuadrado al expresión (\ref{p_def}) y desarrollando dicho cuadrado tenemos
    \begin{equation}
        p^2 = \left( 2n + 1 \right)^2 = 4n^2 + 4n + 1 = 4\left( n^2 + n \right) + 1 = 2m + 1
    \end{equation}
    con $m = 2\left( n^2 + n \right) $. En conclusión, si $p$ es impar, $p^2$ también lo es.
    
\end{proof}
\begin{proof}
    Supongamos por absurdo que $\sqrt{2} \in\Q$.

    \begin{equation}
        \exists\, \alpha\in\Q\tq\alpha^2 = 2, \quad \alpha = \frac{p}{q}
    \end{equation}
    \begin{equation}
        \label{p_equal_q}
        \textrm{entonces } \quad 2 = \alpha^2 = \frac{p^2}{q^2}\quad\rightarrow\quad p^2 = 2q^2 \quad\rightarrow\quad p^2\textrm{ es par }\quad\implies\quad p\textrm{ es par,}
    \end{equation}
    \begin{equation}
        \textrm{entonces, con la expresión (\ref{p_equal_q}) se tiene } \quad p = 2n\quad\implies\quad 4n^2 = p^2 = 2q^2
    \end{equation}
    \begin{equation}
        \textrm{simplificando, }\quad 2n^2 = p^2 = q^2, \textrm{entonces }\quad q^2 = 2n^2\textrm{ es par }\quad\implies\quad q\textrm{ es par.}
    \end{equation}
    Esto es absurdo, ya que implica que $p$ y $q$ tienen $2$ como divisor común.
\end{proof}

\newpage
\subsection{Correspondencias y relaciones de orden}
\begin{definicion}
    Dados dos conjuntos $A$ y $B$ llamamos correspondencia entre $A$ y $B$ a un subconjunto $\mathcal{C}$ del producto cartesiano $A\times B$. Si $\left( a, b \right) \in\mathcal{C}\subset A\times B$ diremos que $a$ está en correspondencia con $b$.
\end{definicion}
\begin{definicion}
    Cuando $A = B$, las correspondencias se llaman relaciones, $\mathcal{R}$. Si $\left( a, b \right)\in\mathcal{R} $ diremos que $a$ está relacionado (o en relación) con $b$, y se denota por $a\mathcal{R}b$.
\end{definicion}
\begin{definition}
    Sea $A$ un conjunto dado no vacío y $\mathcal{R}$ una relación binaria definida en $A$, entonces se dice que $\mathcal{R}$ es una \textbf{relación de orden}.
    \begin{itemize}
        \item\textbf{Reflexiva:} Todo elemento de $A$ está relacionado consigo mismo. Es decir, $\forall a\in A$ tenemos que $a\mathcal{R} a$.
        \item\textbf{Antisimétrica:} Si dos elementos de $A$ se relacionan entre sí, entonces ellos son iguales. Es deicr, $\forall x, y\in A,\ x\mathcal{R} y,\ y\mathcal{R} x \implies x = y$.
        \item\textbf{Transitiva:} Si un elemento de $A$ está relacionado con otro, y este otro a su vez se relaciona con un tercero, entonces el primero estará relacionado también con este último. Es decir, $\forall x, y, z\in A,\ x\mathcal{R} y,\ y\mathcal{R}z\implies x\mathcal{R}z$.
    \end{itemize}
\end{definition}
\begin{notacion}
    Una relación de orden $\rela$ sobre un conjunto $A$ puede denotarse como el par ordenado $\left( A, \mathcal{R} \right) $.
\end{notacion}
\begin{definition}
    Si $\forall a, b\in A$ o bien $a\mathcal{R} b$ o bien $b\mathcal{R} a$, $\mathcal{R}$ se llama \textbf{relación de orden total}.
\end{definition}
