\chapter{The $\Z$ and $\Z_n$ rings}
\thispagestyle{noheaders}

\section{Elemental algebraic structures}
In this section we are defining some structures that frequently appear in mathematics. Although they can
seen unnecessary generalizations, they are pretty useful for us to not to prove the same result in different
contexts.

\begin{defi}[Operation]
    Let $A\subset\uset$, an operation in $A$ is a function from $A\times A$ to $\uset$. When its
    image is in $A$ it is said that it is an \textbf{internal composition law}, or that is \textbf{closed}.
\end{defi}

\hide{
For the next four propositions let $*$ be a closed operation on a set $A\subset\uset$, and let $x, y, z\in A$.

\begin{proposition}
    \textbf{(Associative property).} $x*\left( y*z \right) = \left( x*y \right) *z$.
\end{proposition}

\begin{proposition}
    \textbf{(Identity element).} There exists an \textit{identity element} $e\in A\tq e*x = x*e = x,\ \forall
    x\in A$.
\end{proposition}

\begin{proposition}
    \textbf{(Inverse element).} $\forall x\in A$, there exists an \textit{inverse element} $y$ such that
    $x*y = y*x = e$. We write $y = x^{-1}$.
\end{proposition}

\begin{proposition}
    \textbf{(Commutative property).} $x*y = y*x$.
\end{proposition}
}

\begin{defi}[Group]
    A \textit{group}, $G$, is a set in which it is defined a closed operation, let's
    denote it by $*$, such that the \textbf{associative property},
    \begin{equation}
        g * \left( h * f \right) = \left( g * h \right) * f,
    \end{equation}
    is satisfied, there exists an \textbf{identity element},
    \begin{equation}
        \exists\ e\in G\tq \forall g\in G, e * g = g * e = g,
    \end{equation}
    and there exists an \textbf{inverse element},
    \begin{equation}
        \forall g\in G, \exists\ h\in G \tq h * g = g * h = e. \left( h = g^{-1}\right).
    \end{equation}
    Moreover, if $*$ is a \textbf{commutative operation} $\left( g*h = h*g \right)$ it is said that $G$ is
    an \textbf{abelian}, or \textbf{commutative group}.

    \hide{
    that the following properties are satisfied.
    \begin{itemize}
        \item $*$ is associative: $g * \left( h * f \right) = \left( g * h \right) * f$.
        \item There exists an identity element: $\exists\ e\in G\tq \forall g\in G, e * g = g * e = g$.
        \item There exists an inverse element: $\forall g\in G, \exists\ h\in G \tq h * g = g * h = e. \left(
            h = g^{-1}\right) $.
    \end{itemize}
    }
\end{defi}

\begin{defi}[Ring]
    A \textit{ring}, $X$, is a set in which there are defined two closed operations, $\oplus$
    and $\otimes$ (addition and product), that satisfy the following properties:
    \begin{itemize}[itemsep = -2pt]
        \item $X$ is an abelian group with respect to $\oplus$.
        \item $\otimes$ is an associative operation on $X$.
        \item The distributive laws are hold (from the left) $\left( a\oplus b \right) \otimes c = \left( a\otimes c \right) \oplus \left( b\otimes c \right) $ and (from the right) $c\otimes\left( a\otimes b \right) =
            \left( c\otimes a \right) \oplus \left( c\otimes b \right) $.
    \end{itemize}
\end{defi}

\begin{note}
    If $\otimes$ is commutative, it is said that $X$ is a \textit{commutative ring}, and if $\otimes$ has an
    identity element (multiplicative identity) it is said that the ring is \textit{unitary}, or \textit{with
    identity}.
\end{note}

\begin{defi}[Field]
    A \textit{field}, $F$, is a commutative ring with identity such that $F - \{0\}$ is an
    abelian group with respect to the product, $\otimes$.
\end{defi}

In a not so formal way, an abelian group is a set in which we can add or substract (add the inverse),
while in a commutative ring we can also multiply; and in a field, divide (except for $0$); i.e. every
element in a field has a multiplicative inverse.

When it is not clear what operations are being considered in a set, they are usually indicated explicitly
next to set, all inside parenthesis. Thus, for instance $\left( \Z, + \right) $, which is an abelian group
is the set of integers with the addition operation.

\begin{example}
    $\left( \Z, \cdot \right) $ and $\left( \N, + \right) $ are not abelian groups because, for example,
    3 does not have an inverse in any of the two sets.
\end{example}

\hide{
There are two main operations between integer numbers. These are addition and product.

Now, we look at the integers paying attention to the addition operation, $\left( \Z, + \right) $.
\begin{remark}
    Addition is a binary operation: for all $a, b\in\Z\rightarrow\ a + b\in\Z$. We operate two by two elements.
\end{remark}

Because $a + b\in\Z$ we say that this binary operation is \textbf{closed in \Z}.

\begin{proposition}
    \textbf{(Associative property).} $\left( a + b \right) + c = a + \left( b + c \right) ,\ \forall a, b, c\in\Z$
\end{proposition}
\begin{proposition}
    \textbf{(Existence of an indentity element).} There's an integer $\varphi\in\Z \tq a + \varphi = \varphi + a = a,\ \forall a\in\Z$, and that integer is $\varphi = 0$.
\end{proposition}
\begin{proposition}
    \textbf{(Inverse element).} Every element $a\in\Z$ has an additive inverse, meaning that for every $a\in\Z$ there's $a'\in\Z\tq a + a' = a' + a = 0$, and in this case $a' = -a$.
\end{proposition}
\begin{proposition}
    \textbf{(Abelian or conmutative property).} $a + b = b + a,\ \forall a, b\in\Z$
\end{proposition}

Because the addition satisfies the first three properties we say that $\left( \Z, + \right) $ form a \textbf{group}. An because $\left( \Z, + \right) $ also satisfies the commutative property we say that $\left( \Z, + \right) $ form an \textbf{abelian group}.

\begin{definition}
    \textbf{(Group).} In general, if we have a set $X$ with some binary close operation $*$, satisfying the first three properties we say that $\left( X, * \right) $ is a \textbf{group}. In addition, if $\left( X, * \right) $ also satisfies the commutative property we say that $\left( X, * \right) $ is an \textbf{abelian group}.
\end{definition}

\begin{example}
    Let $\R^* = \R\setminus \{0\} $, we have that $\left( \R^*, \cdot \right) $ is an abelian group with identity $1$, multiplicative inverse $r^{-1} = \frac{1}{r}, r\in\R^*$.

    Another example of an abelian group is $\left( \R, + \right) $.
\end{example}

The product on $\Z$ is another binary closed operation as $a\cdot b\in\Z,\ \forall a, b\in\Z$.

\begin{proposition}
    \textbf{(Associative property).} $\left( a \cdot b \right) \cdot c = a \cdot \left( b \cdot c \right) ,\ \forall a, b, c\in\Z$
\end{proposition}

What do we do if we find the two operations in the same line?

\begin{proposition}
    \textbf{(Distributive property).} $a\cdot\left( b + c \right) = ab + ac$ and $\left( b + c \right)\cdot a = ba + ca,\ \forall a, b, c\in\Z $.
\end{proposition}

Because $\left( \Z, +, \cdot \right) $ satisfies the four properties for the addition and the associative for the product as well as the distibutive we say that $\left( \Z, +, \cdot \right) $ is a \textbf{ring}.
}

\begin{example}
    $\left( \N, +, \cdot \right) $ is not a ring. $\left( \N, + \right) $ is a binary closed operation that satisfies the associative property, has $0$ as an indentity element, but it doesn't have an additive inverse for each $n\in\N$. Therefore, $\left( \N, + \right) $ does not form a group.

    However, $\left( \Q, +, \cdot \right), \left( \R, +, \cdot \right) $ and $\left( \C, +, \cdot \right)$
    are all commutative rings with identity.
\end{example}

\hide{
Suppose that we have a set $X$ with two binary closed operations, $*$ and $\circ$. $\left(X, *, \circ  \right) $ is a ring if $\left( X, * \right) $ is an abelian group and if $\left( X, \circ \right) $ satisfies the associative property and $\left( X, *, \circ \right) $ satisfies the distributive property.

If in a ring $\left( X, *, \circ \right) $ we have that there's a multiplicative identity,
and $\forall a, b\in X, ab = ba$, then we say that the ring is commutative with identity.
}

\hide{
In general, a given integer $n\in\Z$ doesn't have a multiplicative inverse. In other words, there's no $m\in\Z$ such that $nm = 1$ for a given $n\in\Z$. Then, what are the integers $n$ for which there's an $m\in\Z$ such that $nm = 1$? Only $1$ and $-1$. The elements having this property are called \textbf{units}.

$\left( \Q, +, \cdot \right) $ is a commutative ring with identity. Any $r\in\Q\setminus \{0\} $ is a unit as if $r\in\Z, \frac{1}{r}\in\Q \implies r\cdot \frac{1}{r} = 1$. Because all non-zero elements in $\Q$ are units we say that $\left( \Q, +, \cdot \right) $ is a \textbf{field}.
}

\begin{example}
    $\left( \R, +, \cdot \right) $ and $\left( \C, +, \cdot \right) $ are examples of fields.
\end{example}


\section{Greatest common divisor. Euclid's algorithm}

\begin{theorem}[Division algorithm on $\Z$] \label{3:division-algo}
    Let $a, b\in\Z$ with $b > 0$. Then, there exists unique
    integers $q, r$ (called quotient and remainder) such that
    \begin{equation}
        a = bq + r\quad\textrm{ with } 0\leq r < \abs{b}.
    \end{equation}
\end{theorem}

\begin{proof}
    Let $r$ be least positive value that $a - bq$ takes when $q\in\Z$, then $r < \abs{b}$ ya que si $r \geq
    \abs{b}$, aumentando o disminuyendo (si $b$ es negativo) $q$ en una unidad obtendríamos un valor de $r$
    menor, el cociente y el resto son únicos porque $a = q_1 + r_1, a = bq_2 + r_2$ implica $b\left( q_2 - q_1
    \right) = r_1 - r_2$ lo que contradice que $0 \leq r_1, r_2 < \abs{b}$ excepto en el caso trivial $q_2 - q_1
    = r_1 - r_2 = 0$.
\end{proof}

If in theorem \ref{3:division-algo} $r = 0$, then $b$ divides $a$.

\begin{defi}[Divisibility] \label{3:divisibility}
    Given two integers $a$ and $b$ with $b\neq 0$, we say that $b$ \textit{divides}
    $a$, written $b\mid a$, if there is some $q\in\Z$ such that $a = bq$.
    \begin{equation}
        b\mid a\iff \exists\ q\tq a = bq.
    \end{equation}
    Moreover, we also say that $b$ is a \textit{divisor} of $a$, or that $a$ is a \textit{multiple} of $b$.
    Otherwise, if $b$ does not divide $a$ we write $b \nmid a$.

    \hide{Let $a, b\in\Z$. We say that $a$ \textbf{divides} $b$, if there is some $c\in\Z$ such that $b = ac$.
    Moreover, if $a$ divides $b$, then we also say that $b$ is a multiple of $a$.}
\end{defi}

\hide{
\begin{notation}
    If $a$ divides $b$, then we write $a\mid b$. Otherwise, if $a$ does not divide $b$, then we write
    $a\mid b$.
\end{notation}}

\begin{remark}
    $1$ and $-1$ divide all the integers. In other words, all integers are multiple of both $1$ and $-1$
    (including $0$). If $0$ divides an integer $b$ that means by definition that $b = c\cdot 0$ for some
    $c\in\Z$, and that would imply that $b = 0$. Therefore, $0$ only divides $0$. Now, $0$ is a multiple
    of all the integers as if $a\in\Z\implies a\mid 0$ because $0 = 0\cdot a$.
\end{remark}

Now let's take a look to some properties on divisibility.

\begin{prop}
    If an integer $c$ divides another two integers $a$ and $b$, then $c$ also divides any linear combination
    of $a$ and $b$. In other words,
    \begin{equation}
        \textrm{if } c\mid a\textrm{ and } c\mid b\implies c\mid \alpha a + \beta b,\quad\forall\alpha, \beta
        \in\Z.
    \end{equation}

    \hide{Let $a, b, c\in\Z$, then if $c\mid a$ and $c\mid b\implies c\mid \alpha a + \beta b, \forall\alpha,
    \beta\in\Z$. In other words, if an integer $c$ divides two integers $a$ and $b$, then $c$ also divides
any linear combination of $a$ and $b$.}
\end{prop}

\begin{proof}
    Since $c\mid a\implies\exists\ c'\in\Z\tq a = cc'$, and since $c\mid b\implies\exists\ c"\in\Z\tq
    b = cc"$. If we consider $\alpha, \beta\in\Z, \alpha a + \beta b = \alpha\left(cc'\right) + \beta\left(
    cc"\right) = c\left(\alpha c' + \beta c" \right)\implies c\mid \alpha a + \beta b$.
\end{proof}

\begin{prop}
    Let $a, b, c\in\Z$. If $a\mid b$ and $b\mid c$ then $a\mid c$.
\end{prop}

\begin{proof}
    If $a\mid b\implies b = a\alpha, \alpha\in\Z$ and if $b\mid c\implies c = b\beta, \beta\in\Z$, then
    $c = b\beta = a\alpha\beta\implies a\mid c$.
\end{proof}

\begin{prop}
    Let $a, b\in\Z$. If $a\mid b$ and $b\mid a\implies\abs{a} = \abs{b}$.
\end{prop}

\begin{proof}
    Since $a\mid b\implies b = a\alpha$ for some $\alpha\in\Z$, and since $b\mid a\implies a = b\beta$
    for some $\beta\in\Z$. Then, $b = a\alpha = \left( b\beta \right) \alpha = b\left(\alpha\beta\right)
    \implies b = b\left( \beta\alpha \right)\implies b - b\left( \beta\alpha \right) = 0\implies b\left(
    1 - \left( \beta\alpha \right) \right) = 0$.
    \begin{itemize}[itemsep = -2pt]
        \item If $b = 0\implies a = 0$.
        \item If $b\neq 0\implies 1 - \beta\alpha = 0\implies 1 = \beta\alpha\implies \beta = \alpha = 1$
            or $\beta = \alpha = -1$.
        \item If $\beta = \alpha = 1\implies a = b$.
        \item If $\alpha = \beta = -1\implies a = -b$.
    \end{itemize}
\end{proof}

The most remarkable consequence of (\ref{3:divisibility}) is that we can define the greatest common divisor
and that there's a method to compute it called \textit{Euclid's algorithm}. For this reason, when a ring holds
an analogue property to Theorem (\ref{3:division-algo}) it's said that is an Euclidian domain.

\begin{defi}[Greatest common divisor]
    An integer $d$ is said to be the greatest common divisor of two
    integers $a$ and $b$ if $d > 0$, $d$ divides both numbers and if any other integer $c$ divides $a$
    and $b\implies c\mid d$.
\end{defi}

\begin{prop}[Bézout's identity] \label{bezout-id}
    If $d = \gcd\left( a, b \right) $, then exists $n, m\in\Z$ such that $d = an + bm$. In fact, all the solutions $x, y\in\Z$ to the equation $d = ax + by$ are of the form
    \begin{equation}
        \begin{cases}
            x = n - bt/d \\
            y = m + at/d
        \end{cases}
        \quad\quad\textrm{with } t\in\Z.
    \end{equation}
\end{prop}

\begin{theorem}[Well ordering principle]
    Any non-empty subset of $\N$ has a minimum on $\N$.
\end{theorem}

Now, how do we find the $\gcd\left( a, b \right) $? We can use the following theorem.

\begin{theorem}
    Let $a, b\in\Z$ where at least one of them is non-zero. Then, exists a greatest common divisor $d$
    of $a$ and $b$. Moreover, $d$ can be written as $d = a\alpha + b\beta$ for some $\alpha, \beta\in\Z$.
    In fact, $d$ is the smallest positive number that can be written as a linear combination of $a$ and $b$.
\end{theorem}

\begin{proof}
    If we have two numbers $a$ and $b$, the theorem says that we can find $\gcd\left( a, b \right) $ by
    writting all the possible linear combinations of $a$ and $b$.
    \begin{itemize}
        \item If $b = 0$ then $a\neq 0\implies d = \gcd\left( a, b \right) = \abs{a}$
        \item If $a, b\neq 0$, consider the set $S:= \{a\alpha + b\beta\tq\alpha, \beta\in\Z\} $. Consider
            now $S':=\{a\alpha + b\beta\tq\alpha, \beta\in\Z: a\alpha + b\beta > 0\}\subseteq S$.

            Claim: $S'\neq\O$. By construction, $a, b\in S$ if $a > 0\implies a\in S'$, otherwise $-a > 0$
            and $-a\in S'$. Then, $S'\subseteq\N, S'\neq\O$. Making use of the well ordered principle we
            can ensure that $S'$ has a minimum.

            Suppose we call $d = \min S'$. In particular, there exists $\alpha, \beta\in\Z\tq d = a\alpha
            + b\beta$ (by definition, $d$ is the smallest positive number that can be written in this form).

            Claim: $d = \gcd\left( a, b \right) $. By construction we have that $d > 0$. Now, we divide
            $a$ by $b$ using the division algorithm on $\Z\implies\exists\ q, r\in\Z\tq a = qd + r$ where
            $0\leq r < d$. From this we can solve for $r$.
            \begin{equation}
                r = qd - a = q\left( a\alpha + b\beta \right) - a = b\beta + a\left( q\alpha - 1 \right)
            \end{equation}
            Therefore, $r$ can be written as a linear combination of $a$ and $b$. If $r > 0\implies r\in S'
            \implies d\leq r$. Here we reach a contradiction unless $r = 0$.

            In the same way, using the same argument we get that $d\mid b$.

            Now let $c\in\Z: c\mid a$. Since $c\mid a\implies a = ca'$ for some $a'\in\Z$, and since
            $c\mid b\implies b = cb'$ for some $b'\in\Z$, we know that $d = a\alpha + b\beta = \left(
            ca'\right)\alpha + \left( cb'\right)\beta = c\left( a'\alpha + b'\beta \right) = d\implies
            c\mid d$. Therefore, $d = \gcd\left( a, b \right)$.
    \end{itemize}
\end{proof}

\begin{example}
    Let $a = 2$ and $b = 3$, then $\gcd\left( 2, 3 \right) = 1$. The previous theorem basically tells us
    that we can write the $\gcd\left( 2, 3 \right) $ as the linear combination $1 = \left( -1 \right)2 +
    3\cdot 1$. Because of the properties previously seen we know that the linear combination $2\alpha +
    3\beta$ is always divisible by 2.
\end{example}

In order to find the linear expression $d = an + bm$ for given $a, b$ we use the euclidean algorithm.

\hide{
\begin{example}
    We wanna find $d = \gcd\left( 802, 384 \right) $.
\end{example}
}

\begin{prop}
    If $c\mid a$ and $c\mid b\implies c\mid\gcd\left( a, b \right) $. In other words, $\gcd\left( a, b \right)$
    is a multiple of any other common divisor of $a$ and $b$.
\end{prop}

\begin{proof}
    Let $d = \gcd\left( a, b \right)$. Suppose that $c\mid a, b$. Because $d = \gcd\left( a, b \right)\implies
    \exists\ \alpha, \beta\in\Z\tq d = \alpha a + \beta b$. Since $c\mid a\implies a = a'c$ for some $a'\in\Z$,
    and since $c\mid b \implies b = b'c$ for some $b'\in\Z$, then $d = \alpha a + \beta b = \alpha a'c +
    \beta b'c = c\left( \alpha a' + \beta b' \right) $.
\end{proof}

\begin{remark}
    This is just a property of the $\gcd\left( a, b \right)$.
\end{remark}

\begin{example}
    Let $a = 12$ and $b = 30$, then $\gcd\left( a, b \right) = 6\implies $ any common divisor of 12 and 30
    divides 6. Common divisors of 12 and 30 are 1, 2, 3 and 6. In this case, $1, 2, 3\mid 6$ but, for instance,
    2 does not divide 3.
\end{example}

\begin{prop}
    If $\gcd\left( a, b \right) = d$ and if we write $a = da', b = db'$ for some $a', b'\in\Z$, then
    $\gcd\left( a', b' \right) = 1$.
\end{prop}

\begin{proof}
    Since $d = \gcd\left( a, b \right) \implies d = a\alpha + b\beta = \left( a'd \right) \alpha + \left(
    b'd\right)\beta = \left( a'\alpha + b'\beta \right) d$. Now, dividing $d = \left( a'\alpha + b'\beta
    \right)d$ by $d$ we have $1 = a'\alpha + b'\beta\implies \gcd\left( a', b' \right) = 1$.
\end{proof}

\begin{prop}
    Let $a, b, n\in\Z$. Then, $\gcd\left( na, nb \right) = n\gcd\left( a, b \right) $.
\end{prop}

\begin{proof}
    Let $d = \gcd\left( a, b \right) $. Since $d\mid a, b\implies nd\mid na$ and $nd\mid nb\implies
    nd$ is a common divisor of $na$ and $nb$, but is $nd = \gcd\left( na, nb \right)$?

    Again, since $d = \gcd\left( a, b \right) \implies d = a\alpha + b\beta$ for some $\alpha, \beta\in\Z$.
    If we multiply by $n$ we have $nd = na\alpha + nb\beta = \gcd\left( na, nb \right) $.
\end{proof}

\begin{remark}
    Two numbers $a$ and $b$ can have many common divisors, but among them the only one that can be
    written as a linear combination of $a$ and $b$ is the $\gcd\left( a, b \right) $.
\end{remark}

\begin{theorem}[Euclid's theorem]
    Let $a, b\in\Z$. If $a\mid bc$ and $\gcd\left( a, b \right) = 1\implies
    a\mid c$.
\end{theorem}

\begin{proof}
    Since $\gcd\left( a, b, c \right) = 1\implies\gcd\left( ca, cb \right) = c\implies$ since $a\mid ca$ and
    $a\mid bc\implies a\mid c$.
\end{proof}

\begin{example}
    Let $a = 3, b = 4, c = 6$. Then $3\mid bc = 24$. Since $a = 3$ does not divide $b = 4$, then $ a = 3\mid
    c = 6$.
\end{example}

\section{Prime and coprime numbers. Factorization theorem}
\begin{defi}[Coprimes]
    Let $a, b\in\Z$If $\gcd\left( a, b \right) = 1$ we say that $a$ and $b$ are
    \textbf{relatively prime} or \textbf{coprime}.
\end{defi}

\begin{defi}[Prime number]
    An integer $p$ is said to be \textbf{prime} if $p > 0$ and the only divisors
    of $p$ are $\pm 1$ and $\pm p$.
\end{defi}

\begin{prop}
    Every positive integer greater than 1 is divisible by a prime number.
\end{prop}

\begin{proof}
    Let $a\in\Z_{>1}$.
    \begin{itemize}[itemsep = -2pt]
        \item If $a$ is a prime number then the proof is done.
        \item If $a$ is not a prime number, then there exists by definition a $c\in\Z_{>1} < a$ such
        that $c\mid a$.

        Let $S:=\{c\in\Z_{>1} : c\mid a \textrm{ and } c < a\}, S\subseteq\N, S\neq\O$. By the wll order
        principle we know that there is a minimum element in $S$. So, let $d:=\min \{c\tq c\in S\} $. Then,
        $d > 1, d < a$ and $d\mid a$. If $d$ is the smallest integer in $S$ satisfying these three properties
        it has to be a prime number.

        If $d$ was not a prime $\implies\exists b > 1, b < d, b\mid d$, but since $d\mid a\implies b\mid a
        \implies b\in S$, and this is a contradiction. Therefore, $d$ is prime and $d\mid a$.
    \end{itemize}
\end{proof}

\begin{theorem}[Euclid's theorem]
    There are infinitely many primes.
\end{theorem}

\begin{proof}
    Suppose, to get a contradiction, that there is a finite number of primes. Let $a:= 1 + p_1 + \ldots +
    p_n\in\Z_{>1}$. By the claim, $a$ is divisible by some prime. Hence, $\exists\ i\in \{1\ldots n\} :
    p_i\mid a\implies p_i\mid\left( 1 + p_1 + \ldots + p_n \right)\implies 1 = a - p_1\cdot\ldots\cdot p_n
    \implies p_i\mid 1$, which is a contradiction.
\end{proof}

\begin{theorem}[Fundamental theorem of arithmetic]
    Every positive integer $n\in\Z_{>1}$ can be written as
    a product of prime numbers, and this factorization is unique except for the order of the factors.
\end{theorem}

\begin{proof}
    To prove this theorem will are using mathematical induction. Then, the first case that makes sense
    for the statement of the theorem is $n = 2$, which is a prime. That means this is already the
    factorization in product of primes.

    Induction hypothesis: Assume that the theorem holds for positive integers $< n$, for $n > 2$. Using
    the induction hypothesis we'll prove that the theorem also holds for $n \left( n > 2 \right) $. Now,
    \begin{itemize}[itemsep = -2pt]
        \item If $n$ is a prime there's nothing to prove.
        \item Suppose $n$ is not a prime. We have proven that any integer greater than 1 is always
            divisible by a prime. Therefore, there is some prime $p$ that divides $n$: $n = pn_1$,
            with $p > 1$. We know that $n_1 < n$. We can apply the induction hypothesis to $n_1$:
            \begin{equation}
                n_1 = p_1\ldots p_s\implies n = p\cdot p_1\ldots p_s
            \end{equation}
            is a factorization of $n$ into primes.

            So we have proven that any positive number $n > 1$ can be written as a product of primes.
            Now we want to show that this factorization is unique. We will prove this by contradiction.

            Suppose that for a given $n > 1$ we have two factorizations into primes:
            \begin{equation}
                n = p_1\cdot p_2\ldots p_s = q_1\cdot q_2\ldots q_\ell
            \end{equation}
            where $p_i, q_j$ are primes from $i = \ell\ldots s, j = 1\ldots\ell$. Since $p_1$ is a prime
            and since $p_1\mid n\implies p_1\mid q_1\cdot \ldots\cdot q_\ell\implies p_1\mid q_j$ for some
            $j\in \{1\ldots\ell\} $ (Whenever a prime divides a product it has to divide one of the factors).
            So we can cancel $p_1$ and $q_j$ in the expression for $n$ and repeat to conclude that $s = \ell$
            and $\{p_1\ldots p_s\} = \{q_1\ldots q_\ell\} $.
    \end{itemize}
\end{proof}

%\newpage
\section{Linear Diophantine equations}
\begin{defi}[Diophantine equation]
    A \textit{Diophantine equation} is a polynomial equation whose solutions
    are restricted to integers.
\end{defi}

\begin{defi}[Linear Diophantine equation]
    A \textit{linear Diophantine equation} is a first-degree
    Diophantine equation.
\end{defi}

    These type of equations are named after the ancient Greek mathematician Diophantus, and they are
    important when a problem requires a solution in whole amounts.

The study of problems that require integer solutions is often referred to as \textit{Diophantine analysis}.
Although the practical applications of Diophantine analysis have been somewhat limited in the past, this
kind of analysis has become much more important in the digital age, as it is very important in the study
of public-key cryptography, for example.

\begin{prop} \label{3:diophantine-solutions-condition}
    A Diophantine equation of the form $ax + by = c$ with $a, b, c\in\Z$ has integer solutions if and only
    if $\gcd\left( a, b \right) \mid c$. In other words, there's no integer solutions if $\gcd\left( a, b
    \right)$ does not divide $c$.
\end{prop}

\begin{remark}
    The previous proposition is just a consequence of the Bézout's identity.
\end{remark}

\begin{example}
    For instance, the Diophantine equation $6x + 20y = 7$ has no integer solutions as $\gcd\left( 6, 20
    \right) = 2$ but 2 does not divide 7.
\end{example}

\subsection{Initial solution to a linear Diophantine equation}
Finding solutions to linear Diophantine equations involves finding an initial solution, and then altering
that solution is some way to find the remaining solutions. When finding this initial solution is important
to recognize first if the equation we are dealing with has or not solutions in $\Z$. As it is stated in
proposition \ref{3:diophantine-solutions-condition}, one can determine if solutions exist or not by
computing the GCD of the coefficients of the variables, and then determining if the constant term can be
divided by that GCD.

If solutions do exist, then there is an efficient method to find an initial solution. The extended version
of the Euclidean algorithm seen previously will give us both the GCD of the coefficients and an initial
solution.

So, given an equation $ax + by = n$, we will use the Euclidean algorithm to compute $\gcd\left( a, b \right)
= d$ and determine if there are any solutions. The extended version of this algorithm consists on solving
the equations used to compute the GCD for the remainders, and using substitution, go through the steps of
the Euclidean algorithm to find a solution to the equation $ax_i + by_i = d$. Then, the initial solution
to the equation $ax + by = n$ is the ordered pair
\begin{equation}
    \left( x_i\cdot \frac{n}{d}, y_i\cdot \frac{n}{d} \right).
\end{equation}

\begin{example}
    \textbf{Find a solution to the Diophantine equation $6x + 10y = 20$.}

    Since $\gcd\left( 6, 10 \right) = 2$ and $2$ divides $20\implies$ the equation $6x + 10y = 20$ has a
    solution in $\Z$. To make the problem easier we can simplify the equation by dividing by the $\gcd\left( 6,
    10\right) $, which yields
    \begin{equation}\label{3:dio:ex-one:sequation}
        3x + 5y = 10,
    \end{equation}
    which has the exact same solutions as the initial equation. Now, let the equation
    \begin{equation}\label{3:dio:ex-one:one-equation}
        3x_i + 5y_i = 1.
    \end{equation}
    Since $\gcd\left( 3, 5 \right) = 1$, all solutions to (\ref{3:dio:ex-one:one-equation}) multiplied by
    $10$ are solutions to (\ref{3:dio:ex-one:sequation}). In this case, there is no need to use the Euclidean
    algorithm as it is pretty easy to see that $x_i = 2$ and $y_i = -1$ verifies (\ref{3:dio:ex-one:one-equation})
    and therefore, the ordered pair $\left( 20, -10 \right) $ is an initial solution to the equation $6x + 5y
    = 20$.
\end{example}

\subsection{General solution to linear Diophantine equations}

\hide{
\begin{example}
    Find all the integer solutions to $6x + 10y = 20$. We have that $\gcd\left( 6, 10 \right) = 2$ and
    $2\mid 20\implies$ there are solutions. Dividing the equation by 2 we get $3x + 5y = 10$. Both
    equations have the same solutions.

    Now, $\gcd\left( 3, 5 \right) = 1\implies$ there's a linear combination of 3 and 5 that is equal to 1,
    so we can write a new equation $3n + 5n = 1$, which does not have the same solutions as the previous
    but it will help us to find all the solutions. First, we solve our new equation using Euclid's
    algorithm:
    \begin{align}
        5 &= 1\cdot 3 + 2 \\
        3 &= 1\cdot 2 + 1 \\
        1 &= 3 + \left( -1 \right) \cdot 2 = \\
          &= 3 + \left( -1 \right) \left( 5 + \left( -1 \right) \cdot 3 \right) = \\
          &= 2\cdot 3 + \left( -1 \right) \cdot 5
    \end{align}
    Multiplying by 10 we get $10 = \left( 20 \right) \cdot 3 + \left( -10 \right) \cdot 5$. Then, a
    solution to $3x + 5y = 10$ is $x = 20$ and $y = -10$. However, how do we find all the solutions to
    this equation?

    Suppose $\left( x_0, y_0 \right)$ is an integer solution to $3x + 5y = 10$, and suppose that $\left(
    x_1, y_1\right) $ is another solution. Subtracting both equations we get
    \begin{equation}
        3\left( x_0 - x_1 \right) + 5\left( y_0 - y_1 \right) = 0 \iff 3\left( x_0 - x_1 \right) =
        5\left( y_1 - y_0 \right)
    \end{equation}
    Since $3\mid 5\left( y_1 - y_0 \right)$ and knowing $\gcd\left( 3, 5 \right) = 1, 3\mid \left( y_1
    - y_0 \right) \implies y_1 - y_0 = 3\ell$ for some $\ell\in\Z$. Substituting, $3\left( x_0 - x_1
    \right) = 5\cdot 3\ell\implies 5\mid\left( x_0 - x_1 \right) $, in fact we get that $x_0 -  x_1 = 5\ell$.
    Once a solution to the equation is found, $\left( x_0, y_0 \right) $, the other solutions are as follows:
    $\left( x_1, y_1 \right) $ where $x_1 = x_0 - 5\ell$ and $y_1 = y_0 + 3\ell$ for some $\ell\in\Z$.
\end{example}
}

\section{Congruences}
\hide{
\begin{definition}
    \textbf{(Congruence).} Given $n\in\N$ and $a, b\in\Z$ we say $a$ is \textit{congruent} to $b$ modulo $n$
    and we write $a\equiv b \left( \mod n \right) $ if $n\mid a - b$.
\end{definition}

\begin{example}
    For instance, $8\equiv 3 \left( \mod 5 \right) $ as $5\mid 8 - 3$.
\end{example}

\begin{lemma}
    Two numbers are congruent modulo $n$ if and only if when divided by $n$ they leave the same remainder.
\end{lemma}

\begin{proposition}
    A congruence $\equiv \left(\mod n \right) $ is an equivalence relation on $\Z$.
\end{proposition}
}

If we fix an integer $n\in\Z$ then we can define an equivalence relation $\sim$ on $\Z$ by saying that
$x\sim y \iff x - y$ is a multiple of $n$; i.e. if $n\mid x - y$.

\begin{defi}[Congruence]
    Let $n\in\N$, $a$, $b\in\Z$ and let $\sim_n$ be an equivalence relation on $\Z$
    such that $a\sim_n b \iff n\mid a - b$. If $a\sim_n b$ it's said that $a$ and $b$ are \textit{congruent
    modulo $n$}, denoted by $a\equiv b \mod n$.
\end{defi}

\hide{
\begin{example}
    $2\equiv 7\mod 5$, $-13\equiv 21\mod 17$, $90\equiv 126\mod 6$.
\end{example}
}

\begin{lemma}
    Two numbers are congruent modulo $n$ if and only if when divided by $n$ they leave the same remainder.
\end{lemma}

\begin{proof}
    Suppose
    \begin{equation}
        a = nq_1 + r_1\quad\quad b = nq_2 + r_2
    \end{equation}
    Then, $a\equiv b\mod n\implies \left( a - nq_1 \right)  - \left( b - nq_2 \right) \implies n\mid r_1 - r_2$,
    and as $0 < r_1, r_2 < \abs{n}$, this implies $r_1 = r_2$.
\end{proof}

\begin{defi}
    For $x\in\Z$ it is defined the equivalence class of $x$ with respect to the congruece equivalence
    relation $\equiv \mod n$ by
    \begin{equation}
        [x] \bydef \{a\in\Z\tq a\equiv x\mod n\}
    \end{equation}
\end{defi}

\begin{example}
    Take $n = 3$ and $x = 0$, $1$, $2$. So this yields the equivalence classes
    \begin{align}
        [0] &= \{a\in\Z\tq a\equiv 0\mod 3\} = \{ 0, \pm 3, \pm 6, \ldots\} \\
        [1] &= \{a\in\Z\tq a\equiv 1\mod 3\} = \{\ldots, -5, -2, 1, 4, 7, 10, \ldots\} \\
        [2] &= \{a\in\Z\tq a\equiv 2\mod 3\} = \{\ldots, -4, -1, 2, 8, 11, \ldots\}
    \end{align}
\end{example}

From the previous example we work out that $\equiv\mod n$ divides $\Z$ in $n$ equivalence classes or partitions that correspond to the $n$ possible values of the remainder ($r = 0, 1, 2,\ldots,\abs{n} - 1$).

\begin{defi}
    Fixed $n$, the set of least residues is given by $\{0, 1, \ldots, n - 1\} $.
\end{defi}

Therefore, for all $a\in\Z$, $a$ is congruent to exactly one of the least residues modulo $n$.

\begin{proof}
    Use the division algorithm with $a$ and $n$. This led us to
    \begin{equation}
        a = nq + r\quad\textrm{ with } 0\leq r \leq n - 1.
    \end{equation}
    From this it follows that $a - r = nq\implies a\equiv r\mod n$.
\end{proof}

\begin{prop}
    If $a\equiv b\mod n$ and $c\equiv d\mod n$ then $a + c\equiv \left( b + d \right) \mod n$ and $ac\equiv
    bd\mod n$.
\end{prop}
% This proof can be found here: youtu.be/33p04rolbHA

\begin{prop}
    If $a\equiv b\mod m$ and $n\mid m$ then $a\equiv b\mod n$.
\end{prop}
% Proof can also be found in the previous video

\hide{
\begin{example}
    Si $n = 5$ the equivalence classes are
    \begin{align}
        [0] = \{\ldots, -5, 0, 5, 10, 15, \ldots\} \\
        [1] = \{\ldots, -4, 1, 6, 11, 16, \ldots\} \\
        [2] = \{\ldots, -3, 2, 7, 12, 17, \ldots\} \\
        [3] = \{\ldots, -2, 3, 8, 13, 18, \ldots\} \\
        [4] = \{\ldots, -1, 4, 9, 14, 19, \ldots\}
    \end{align}
\end{example}
}

\begin{defi}
    We say that $\Z_n$ is the quotient set of the relation of congruence modulo $n$ by $\Z$.
\end{defi}

\begin{prop}
    If we define in $\Z_n$ the sum and multiplication operations
    \begin{equation}
        [a] + [b] = [a + b]\quad\textrm{ and }\quad[a]\cdot[b] = [a\cdot b]
    \end{equation}
    then, in general, $\left( \Z, +, \cdot \right)$ is a commutative ring, as both operations have same properties they
    have on $\Z$.
\end{prop}

\hide{
Note that, at first sight, these operations are not clearly well defined, as we can choose different
representatives of the same equivalence class. For instance, in $\Z_5$ we have that $[2] = [7]$ and $[4] =
[19]$
}

\begin{prop} \label{3:zn-field-condition}
    $\left( \Z_n, +, \cdot \right) $ is a field $\iff n$ is prime.
\end{prop}

\begin{proof}
    If $n$ is not prime, $n = ab$ with $0 < \abs{a}, \abs{b} < \abs{m}$, then $[a], [b]\neq [0]$ and
    $[a]\cdot[b] = [m] = [0]$. Therefore, $[b]$ can't have a multiplicative inverse because
    \begin{equation}
        [b]\cdot[c] = [1] \implies [a]\cdot[b]\cdot[c] = [0]
    \end{equation}
    and this contradicts our hypothesis about $a$. Then, $\Z_n$ is not a field.

    On the other hand, if $n$ is prime, then any $a$ with $1 \leq a < p$ holds that $\mcm\left( a, p \right)
    = \pm 1$, and this implies that exists some integers $x$ amd $y$ such that $ax + py = 1$. Once we know
    this integers we have
    \begin{equation}
        [ax + py] = [1]\implies [ax] + [py] = [1]\implies [ax] = [1].
    \end{equation}
    Therefore, the class $[a]$ has an inverse (which is $[x]$).
\end{proof}

\begin{prop} \label{3:class-inverse-condition}
    In general, an equivalence class $[a]$ has a multiplicative inverse in $\Z_n \iff a$ and $n$ are coprimes; i.e. $\gcd\left( a, n \right) = 1$.
\end{prop}

\begin{proof}
    Since $\gcd\left( a, n \right) = 1$, and applying the Bezout's identity (\ref{bezout-id}), we have that
    $\exists m, \ell\in\Z\tq am + b\ell = 1\iff am\equiv 1\mod n$.
\end{proof}

\begin{notation}
    Usually we write $\Z^*_n$ to design the set of equivalence classes of $\Z_n$ that have a multiplicative
    inverse.
\end{notation}

\noindent Proposition (\ref{3:class-inverse-condition}) has several consequences:
\begin{itemize}
    \item Let $p$ be a prime number. Then, we can define the set
        \begin{equation}
            \Z^*_p := \{[1], \ldots, [p - 1]\}
        \end{equation}
        in which all elements but zero have a multiplicative inverse; i.e. they are units. Therefore, by
        (\ref{3:zn-field-condition}) $\left( \Z_p, +, \cdot \right) $ is always a field.
    \item Let $n\in\Z_{> 0}$, which is not a prime number. Then, we can define the set
        \begin{equation}\label{3:zns-def}
            \Z^*_n := \{[a]\in\Z_n\tq \gcd\left( a, n \right) = 1\}
        \end{equation}
        which contains the equivalence classes that have a multiplicative inverse in $\Z_n$. Now, we will
        use the \textit{Euler's $\varphi$ function} to represent the cardinality of (\ref{3:zns-def}).
        \begin{equation}
            \varphi\left( n \right) := \abs{\Z^*_n} = \card \left( \Z^*_n \right)
        \end{equation}
        A question arise here. What are the values of $\varphi\left( n \right) $? If $n = p$, where $p$ is
        a prime, we have that $\varphi\left( n \right) = \abs{\Z^*_p} = p - 1$.

        However, if $n = p^k$, where $p$ is prime and $k\in\Z_{\geq 1}$, to find a value of $\varphi
        \left( p^k \right) $ we should write a list of the elements in $\Z_{p^k}$ and remove the classes
        that come from multiples of $p$. In this case, there are $p^{k-1}$ elements that come from multiples
        of $p$. Thus, we get $\varphi\left( p^k \right) = p^k - p^{k-1}$.

        This result let us know how many elements in a ring with $n = p$ have a multiplicative inverse.
        Together with the following lemma we can determine, for instance, $\abs{\Z^*_{12}}$.
\end{itemize}

\begin{lemma}\label{3:divide-euler-phi}
    Suppose $n$ and $m$ in $\Z$ are coprime. Then $\varphi\left( nm \right) = \varphi\left( n \right) \cdot
    \varphi\left( m \right) $.
\end{lemma}

\begin{example}
    To compute $\abs{\Z^*_{12}}$ we know that $12 = 3\cdot 4$ and $\gcd\left( 3, 4 \right) = 1$. Then,
    by lemma \ref{3:divide-euler-phi} we get
    \begin{equation}
        \varphi\left( 12 \right) = \varphi\left( 3 \right) \cdot \varphi\left( 4 \right) = 2\cdot \left(
        2^2 - 2\right) = 4.
    \end{equation}
\end{example}

\begin{prop}[Cancelation law]
    For $a$, $b$, $c\in\Z$ we have $ca\equiv cb \mod n \iff a\equiv b\mod \frac{n}{\gcd\left( n, c \right) }$.
\end{prop}

\begin{proof}
    Let us prove $\implies$ first. Suppose $ca\equiv cb\mod n\implies n\mid c\left( a - b \right) \implies c\left( a - b \right) = nk, k\in\Z$. Let $d = \gcd\left( c, n \right) $. Note that $\frac{c}{d}\left( a - b \right) = \frac{n}{d}k$, then $\gcd\left( \frac{c}{d}, \frac{n}{d} \right) = 1\implies a - b = \frac{n}{d}k', k'\in\Z$.
    Finally, $a\equiv b\mod \frac{n}{d}$. The proof for the other direction is pretty easy.
\end{proof}

\hide{
\begin{proposition}
    \textbf{(Cancelation law).} In an expression of the form $[a][x]\equiv[a][y]\mod n$, we can conclude
    that $[x]\equiv[y]\mod n \iff [a]$ has a multiplicative inverse in $\Z$.
\end{proposition}
}

\begin{theorem}[Euler's theorem]
    Let $a, n\in\Z$. If $\gcd\left( a, n \right) = 1$, then $[a]^{\varphi\left(
    n\right) }\equiv[1]\mod n$.
\end{theorem}

\begin{proof}
    To be done.
\end{proof}

\hide{
\begin{example}
    Take $\Z^*_{10} = \{[1], [3], [7], [9]\}, \varphi\left( n \right) = 4$.

    Then $[3]^4\equiv [1]\mod 10$ as $[3][3][3][3] = [9][9] = [1]\mod 10$.

    Now take $[2]\not\in \Z^*_{10}$, then $[2]^4\equiv[6] \notequiv [1]$.
\end{example}
}

\subsection{Linear congruences}
%Consider an equation of the form $[a]$

\begin{theorem}
    Consider an equation of the form $[a][x]\equiv[b]\mod n$. Then,
    \begin{itemize}[itemsep = -2pt]
        \item if $\gcd\left( a, n \right) = d \nmid b\implies$ the equation has no solutions.
        \item if $\gcd\left( a, n \right) = d\mid b\implies$ the equation has $d$ different solutions in
            $\Z^*_n$.
    \end{itemize}
\end{theorem}

\begin{proof}
    The equation $[a][x]\equiv[b]\mod n$ has a solution if and only if $ax + ny = b$ has integer solutions,
    and this happens when $\gcd\left( a, n \right) \mid b$. This proves the first statement of the theorem.

    Now, let's prove the second statement. If $\gcd\left( a, n \right) = d\mid b$, then to solve $[a][x]
    \equiv[b]\mod n$ in $\Z_n$ we look for the integer solutions to $ax + ny = b$, which are of the form
    \begin{equation}
        \begin{cases}
            x = x_0 + \frac{n}{d}\ell \\
            y = y_0 - \frac{a}{d}\ell
        \end{cases}\quad\textrm{ with } \ell\in\Z,
    \end{equation}
    where $\left( x_0, y_0 \right) $ is a particular solution. This gives us that the solutions to $[a][x]
    \equiv[b]\mod n$ are of the form
    \begin{equation}
        [x] = [x_0] + \frac{[n]}{d}[\ell]\mod n.
    \end{equation}
    Taking $\ell = 0, 1, 2, \ldots, d - 1$ gives us $d$ different solutions in $\Z_n$.
\end{proof}
