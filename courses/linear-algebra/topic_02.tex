\chapter{Equivalence relations}
In mathematics sometimes there are situations in which it is convenient to establish relations among the
elements of a set. Defining a relation on a set means we have a way to compare the elements in that set.
Under certain conditions (equivalence relations) this will one to subdivide the elements of a set into
different groups that share similar properties. \cite{chamizo02}

\begin{defi}[Relation]
    A relation on a set $A$ is a non-empty subset, $\rela$, of $A\times A$. If $x, y\in A$ holds that
    the ordered pair $\left( a, b \right)\in\rela$, it is said that $a$ \textbf{is related to} $b$, usually
    written $a\rela b$.
\end{defi}

So technically any subset of $A\times A$ is a relation on $A$. However, it should posses certain
characteristics in order to be kind of interesting. Relations are generally used to compare two elements in
some way. That is, we use them to determine whether two elements are \textit{related} in the manner specified.

\begin{example}
   Let the relation $x\rela y\iff x\tq y$ with $x, y\in\N$. Then, that relation is defined by the following
   set:
   \begin{equation}
       \rela = \{n, m : n\mid m\}.
   \end{equation}
\end{example}

\begin{remark}
    For any set $A$, both $\O$ and $A\times A$ are relations on $A$. The $\O$ relation doesn't relate elements to anything, not even themselves. On the other hand, the relation $A\times A$ relates every element to every element of $A$. Actually, this two relations are pretty useless, but worth mentioning.
\end{remark}

\begin{defi} \label{2:relation-props}
    Let $\rela$ be a relation defined on a set $A$. Then, $\rela$ is
    \begin{itemize}
        \item\textbf{Reflexive,} if $\forall x\in A$, $x\rela x$.
        \item\textbf{Symmetric,} if $\forall x, y\in A$, $x\rela y\implies y\rela x$.
        \item\textbf{Antisymmetric,} if $\forall x, y\in A$, $x\rela y$ and $y\rela x\implies x = y$.
        \item\textbf{Transitive,} if $\forall x, y, z\in A$, $x\rela y$ and $y\rela z\implies x\rela z$.
    \end{itemize}
\end{defi}

%\newpage

\section{Equivalence and order relations}
Taking Definition \ref{2:relation-props} into account, is convenient to point out two main types of
relations.

\begin{defi}[Equivalence relations]
    An \textit{equivalence relation}, denoted by $\sim$, is a relation that
    holds the \textbf{reflexive}, \textbf{symmetric} and \textbf{transitive} properties.
\end{defi}

\begin{defi}[Order relation]
    An \textit{order relation} is relation that holds the \textbf{reflexive},
    \textbf{antisymmetric} and \textbf{transitive} properties. Moreover, if $\forall x, y$ is hold $x\rela y$
    or $y\rela x$, then it is said that it is a \textbf{total order relation}. Otherwise, it is said that it
    is of \textbf{partial order}.
\end{defi}

\begin{example}
    \textbf{Prove or disprove that $x\rela y\iff x\leq y$ with $x, y\in\N$ is an equivalence relation.}

    So, in order to check if this is an equivalence relation we should first check if it satisfies the
    reflexive, symmetric and transitive properties.

    \begin{itemize}
        \item[(a)] $x\rela x \iff x\leq x,\ \forall x\in\N$, then the relation $\rela$ is reflexive.
        \item[(b)] If $x\rela y$ then $y\rela x$? No. Just take $2\rela 4 \iff 2\leqslant 4$ but $4\rela 2
            \iff 4\nleqslant 2$. Then, the relation is not symmetric, and therefore it is not an equivalence
            relation.
        \item[(c)] If $x\rela y$ and $y\rela z \implies x\rela z$?
            \begin{equation}
                \sysdelim.\}\systeme{x\rela y \quad\implies\quad x\leqslant y,y\rela z \quad\implies\quad y\leqslant z}
                \quad\implies\quad x\leqslant z \iff x\rela z\quad\implies\quad\rela\textrm{ is transitive.}
            \end{equation}
    \end{itemize}
    Because the reflexive and the transitive conditions are met, but not the symmetric, the $\rela$ relation is not an equivalence relation.
    \qed
\end{example}

\section{Equivalence classes and the quotient set}
Sometimes when working with elements in a set, it is convenient to consider that some of them are \textit{equal}, even though they are not. To declare two elements as equal we define a relation in the set and, to make sure that we don't get something ilogical, we need this relation to be an equivalence relation.

\begin{defi}[Equivalence class]
Given an equivalence relation $\sim$ on a set $A$ and an element $a\in A$, the \textbf{equivalence class} of $a$ is the set
    \begin{equation}
        [a] = \overline{a} \bydef \{x\in A \tq a\sim x\}
    \end{equation}
\end{defi}

In other words, we can say that the equivalence class of the element $a$ of a set $A$ is the set of all
elements $x$ in the set $A$ which are related to the element $a$. We should also point out that the
equivalence classes correspoding to non-related elements are disjoint and non-empty. Therefore, they
define a partition of $A$.

\begin{defi}[Quotient set]
    The quotient set of a set $A$ by the equivalence relation $\sim$, denoted by $A / \sim$, is the set of all the equivalence classes.
\end{defi}

\begin{example}
    Let $\sim$ be an equivalence relation defined on the set $A = \{1, 2, 3, 5, 6, 9\} $ such that $n\sim m
    \iff 3\mid n - m$ (3 divides $n - m$), $n, m\in\Z$. Then, we have
    \begin{equation}
        [1] = \{1\}\quad\quad [2] = [5] = \{2, 5\}\quad\quad [3] = [6] = [9] = \{3, 6, 9\}.
    \end{equation}
    Therefore, we can write the quotient set as $A / \sim = \{[1], [2], [3]\} $.
\end{example}

\hide{
\begin{example}
    Let $x\sim y$ be an equivalence relation between $x$ and $y$ if $3\tq \left( x - y \right)$. Now, we can think about a new set where
    \begin{align}
        0&\equiv 3\equiv -3\equiv 6\equiv 9\equiv\ldots\quad\rightarrow\quad\overline{0} \\
        1&\equiv 4\equiv 7\equiv 13\equiv\ldots\quad\rightarrow\quad\overline{1} \\
        2&\equiv 5\equiv 8\equiv\ldots\quad\rightarrow\quad\overline{2}
    \end{align}
    \begin{equation}
        \Z / \sim = \{[0], [1], [2]\}
    \end{equation}
    %This is known as  \textbf{quotient set}.
\end{example}
\begin{example}
    Let $x\rela y$ be a relation between $x$ and $y$ if $\abs{x} = \abs{y}$ where $x,y\in\R$. Is this an equivalence relation?
    \begin{itemize}
        \item[(a)] $x\rela x\iff \abs{x} = \abs{x}$. Then, the relation $x\rela y$ is reflexive $\forall x\in\R$.
        \item[(b)] $x\rela y\implies \abs{x} = \abs{y}\implies y\rela x$. Then, the relation $x\rela y$ is symmetric because $\abs{y} = \abs{x}$.
        \item[(c)] If $x\rela y$ and $y\rela z\implies x\rela z$?
            \begin{equation}
                \sysdelim.\}\systeme{\textrm{Since } x\rela y\quad\implies\quad \abs{x} = \abs{y},\textrm{Since } y\rela z\quad\implies\quad \abs{y} = \abs{z}}
                \quad\implies\quad \abs{x} = \abs{z}\quad\implies\quad x\rela z.
            \end{equation}
            Then, the relation $x\rela y$ is transitive.
    \end{itemize}
Therefore, it is an equivalence relation.
\begin{align}
    \overline{1} &= \{r\in\R\tq 1\rela r\} = \{1, -1\} \\
    \overline{0} &= \{0\} \\
    \overline{-\sqrt{2}} &= \{\sqrt{2}, -\sqrt{2}\} \\
    \vdots \\
    \R / \R &= \{\overline{r}\tq r\in\R\} = \R_{\geq 0}
\end{align}
\begin{itemize}
    \item Each $r\in\R$ is in the same equivalence class as some element in $\R_{\geq 0}$.
    \item Two different reals in $\R_{\geq 0}$ correspond to two different classes.
\end{itemize}
\end{example}
}
%\subsection{Equivalence relations as set partitions}
