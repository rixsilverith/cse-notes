\documentclass[10pt, a4paper, landscape]{article}
\usepackage[landscape]{geometry}
%\usepackage[colorlinks=true, linkcolor=blue!30!black]{hyperref}
\usepackage{titlesec}
\usepackage{calc}
\usepackage{multicol, multirow}
\usepackage[nochapters]{exmath}
\usepackage{ifthen}
\usepackage{longtable}

\ifthenelse{\lengthtest{\paperwidth = 11in}}{\geometry{top=.4in, left=.5in, right=.5in, bottom=.4in}}{
    \ifthenelse{\lengthtest{\paperwidth = 297mm}}
        {\geometry{top=1cm, left=1cm, right=1cm, bottom=1cm}{\geometry{top=1cm, left=1cm, right=1cm, bottom=1cm}}}
}

%\renewcommand\headrulewidth{0pt}
\setlength\footskip{16pt}

\makeatletter
\renewcommand\section{\@startsection{section}{1}{0mm}{-1ex plus -.5ex minus -.2ex}{0.5ex plus .2ex}{\normalfont\large\bfseries}}
\renewcommand\subsection{\@startsection{subsection}{2}{0mm}{-1ex plus -.5ex minus -.2ex}{0.5ex plus .2ex}{\normalfont\normalsize\bfseries}}
\renewcommand\subsubsection{\@startsection{subsubsection}{3}{0mm}{-1ex plus -.5ex minus -.2ex}{1ex plus .2ex}{\normalfont\small\bfseries}}
\makeatother

\setcounter{secnumdepth}{0}
\setlength\parindent{0pt}
\setlength\parskip{0pt plus .5ex}

\title{Electromagnetism 210 Cheetsheet}
\begin{document}

\raggedright
\footnotesize

\begin{center}
\vspace{-50mm}
\Large{\vspace{-15mm}\textbf{Electromagnetism 210 Cheetsheet}}
\vspace{-.4mm}
\end{center}

\begin{multicols*}{3}
\setlength\premulticols{1pt}
\setlength\postmulticols{1pt}
\setlength\multicolsep{1pt}
\setlength\columnsep{2pt}

% ----------------------------------------------------------------------
\section{Review of vector analysis in $\R^3$}
\subsection{Vectors in $\R^3$}
\textbf{Components.} $\vec{v} = \vec{v_x} + \vec{v_y} + \vec{v_z} = v_x\ihat + v_y\jhat + v_z\khat$.

\textbf{Euclidean norm.} $\vnorm{\vec{v}}\bydef\sqrt{\innerp{\vec{v}, \vec{v}}} = \sqrt{v_x^2 + v_y^2 + v_z^2}$.

\textbf{Unit vector.} $\uvec{v}\bydef\vec{v} / \vnorm{\vec{v}} = \cos(\theta_x)\ihat + \cos(\theta_y)\jhat + \cos(\theta_z)\khat$.

\textbf{Inner product.} $\innerp{\vec{v}, \vec{w}}\bydef v_xw_x + v_yw_y + v_zw_z = \vnorm{v}\vnorm{w}\cos(\theta)$.

\textbf{Orthogonality.} $\vec{v}\perp\vec{w}\iff\innerp{\vec{v}, \vec{w}} = 0$.

\textbf{Cross product.} \[
    \vec{v}\times\vec{w}\bydef\begin{vmatrix}
        \ihat & \jhat & \khat \\
        v_x & v_y & v_z \\
        w_x & w_y & w_z
    \end{vmatrix}.
\]
The vector $\vec{x}\times\vec{w}$ is orthogonal to the plane generated by $\vec{v}$ and $\vec{w}$, its direction is given
by the right-hand rule and $\vnorm{\vec{v}\times\vec{w}}\bydef\vnorm{\vec{v}}\vnorm{\vec{w}}\sin(\theta)$.

\subsection{Scalar and vector fields}
\textbf{Scalar field.} Region of space in which we have defined a scalar magnitude for each point as a function of position.

\textbf{Vector field.} Same as a scalar field but with vector magnitudes.

\textbf{Gradient.} The gradient of a function indicates the rate of change of a magnitude in a given direction. It's a vector
whose direction is the one of maximum growth of the function at that point and its magnitude is its slope.
$$\grad f\bydef\partialdv{f}{x}\ihat + \partialdv{f}{y}\jhat + \partialdv{f}{z}\khat$$.

\textbf{Flux of a vector field through a surface.} Number of force lines that pierces the surface. Surfaces are defined by
the value of their area and are assigned a vector $\uvec{s}$ to orthogonal to them. $$\int\vec{F\dd\vec{s}}\bydef\int\vec{F}
\uvec{s}\dd s.$$
The flux is a scalar.

\textbf{Conservative field.} $\vec{E}(\vec{r}) = \grad\varphi(\vec{r})$.

\textbf{Potential energy in conservative fields.} Work is independent of the followed path in conservative fields, so it can
be expressed as the difference of potential energy. $$W = -\Delta U = \grad\vec{F},$$ $$\grad\vec{E} = -\Delta V.$$

\section{Electrostatic field in vacuum}
\subsection{Electric charges and electrostatics}
\textbf{Electric charge.} Fundamental property of matter. It can be either positive or negative.
\begin{itemize}
    \item Charge is preserved in all kind of processes.
    \item Measured in Coulombs (C).
    \item Charge is quantized. $Q = ne$ where $Q$ is the total charge of a body, $n$ the number of charges and $e$ the 
        charge of the electron.
\end{itemize}

\textbf{Coulomb's law.} $\vec{F} = \frac{1}{4\pi\epsilon_0}\frac{Qq}{\vnorm{\vec{r}}^2}\uvec{r} = 
K\frac{Qq}{\vnorm{\vec{r}}^2} \uvec{r}$.

The sign of force given by Coulomb's law is determined by the sign of the product $Qq$. Therefore, the electrostatic force is
atractive for charges of opposite sign and repulsive for charges of the same sign. This force follows both the superposition
and the action-reaction principles.

Coulomb's law describe how electric charges behave at rest.

\textbf{Superposition principle.} Interaction between two charges is independent of the other charges. $\vec{F} = \displaystyle
\sum_{i=1}^n \vec{F_i}$.

\subsection{Electrostatic field}
\textbf{Field.} A field is just the representation of the distribution of some magnitude. The distribution of the 
electrostatic force generated by charges gives place to the electrostatic field.

\textbf{Electrostatic field.} Perturbation of a certain region of space by the presence of an electric charge.

The electrostatic force given by Coulomb's law can be rewritten as $\vec{F}(\vec{r}) = q\vec{E}(\vec{r})$ where $\vec{E}(\vec{r})$
is the electrostatic field generated by a charge $Q$ at the point $\vec{r}$. Then, $$\vec{E}(\vec{r}) = \frac{1}{4\pi\epsilon_0}
\frac{Q}{\vnorm{\vec{r}}^2}\uvec{r}.$$ Note that the field only depends on the source charge.

$\vec{E}$ can also be defined as the electrostatic force per unit charge, and their units are $N/C$.

\textbf{Field lines.} Represents the electrostatic field (or any other field), are lines tangent to the field vector at each
of its points.

For a system of two charges with the same magnitude but different sign, the field lines go from the positive to the negative
charge.
\begin{itemize}
    \item The number of field lines is proportional to the electric charge.
    \item The density of field lines at a point is proportional to the value of the electrostatic field at that point.
    \item Field lines can't intersect each other. Otherwise we would have two different vectors for the field at that point.
\end{itemize}

\textbf{Superposition principle.} $\vec{E}(\vec{r}) = \displaystyle\sum_{i=1}^n\vec{E}(\vec{r}) = \displaystyle\sum_{i=1}^n
\frac{1}{4\pi\epsilon_0}\frac{Q_i}{\vnorm{\vec{r} - \vec{r_i}}^2}\uvec{r_i}$.

\subsection{Electrostatic potential}
Since the electrostatic force is a conservative one, the work done by it is independent of the followed path and can be
computed from a scalar function known as \textbf{electrostatic potential energy}, $U$. $$\vec{F}(\vec{r}) = -\grad U(\vec{r})
\iff U(\vec{r}) = -\int_{\vec{r_0}}^{\vec{r}}\vec{F}\dd\vec{r}.$$

\textbf{Work done by a force.} $W\bydef\displaystyle\int_{\vec{r_0}}^{\vec{r}}\vec{F}\dd\vec{r}$, measured in Jules.
If the vector field is conservative, $W\bydef\innerp{\vec{F}, \vec{r}}\bydef\vnorm{\vec{F}}\vnorm{\vec{r}}\cos(\theta)$, which can
also be expressed as $\grad\times\vec{F} = 0$.

\textbf{Electrostatic potential energy.} $\vec{F}(\vec{r}) = -\grad U(\vec{r})$.

\textbf{Work as the difference of potential energy.} $W = -\Delta U = U(\vec{r_0}) - U(\vec{r})$.

This means that we can compute $U(\vec{r})$ as the work done opposed to $\vec{F}$ from some reference $\vec{r_0}$ (where
arbitrarly $U(\vec{r_0}) = 0$) to $\vec{r}$.

\textbf{Potential energy between two point charges.} $U(\vec{r})\bydef\frac{1}{4\pi\epsilon_0}\frac{Qq}{\vnorm{\vec{r}}}$.

This equation can be derived as
\begin{align*}
    U(\vec{r})&\bydef -\int_{\vec{r_0}}^{\vec{r}}\vec{F}\dd\vec{r} \\
              &= -Q\int_{\vec{r_0}}^{\vec{r}}\frac{1}{4\pi\epsilon_0}\frac{q}{\vnorm{\vec{r}}^2}\dd\vec{r} = K_e
              \frac{Qq}{\vnorm{\vec{r}}} + \underbrace{K_e \frac{Qq}{\vnorm{\vec{r_0}}}}_{(*)} \\
              &= K_e\frac{Qq}{\vnorm{\vec{r}}}.
\end{align*}
(*) is a constant that depends on where we impose that $U =0$. Taking $r_0 = \infty$ is equivalent to saying that.

\textbf{Superposition principle.} $$U(\vec{r})\bydef\sum_{i > j}^n \frac{1}{4\pi\epsilon_0}\frac{Q_iq_j}{\vnorm{\vec{r_{ij}}}}.$$

\textbf{Electrostatic potential.} Electrostatic potential energy per unit charge. $V(\vec{r})\bydef\displaystyle
\frac{U(\vec{r})}{q} = \frac{1}{4\pi\epsilon_0}\frac{Q}{\vnorm{\vec{r}}} =
-\displaystyle\int_{\vec{r_0}}^{\vec{r}}\vec{E}\ \dd\vec{r}$.
\begin{itemize}
    \item The sign of the potential depends on the sign of the source charge.
    \item Measured in Volts (V) $\equiv$ J/C.
\end{itemize}

\textbf{Superposition principle.} $V(\vec{r})\bydef\displaystyle\sum_{i=1}^n V_i(\vec{r}) = \displaystyle\sum_{i=1}^n
\frac{1}{4\pi\epsilon_0}\frac{Q_i}{\vnorm{\vec{r} - \vec{r_i}}}$.

\textbf{Equipotential surfaces.} Surfaces in which the electrostatic potential is constant. Equipotential surfaces created by
a point charge are concentric spheres centered at the charge. $\vec{E}$ is always orthogonal to these surfaces.

\textbf{Potential difference (a.k.a. voltage).} $\dd V(\vec{r}) = \frac{U(\vec{r})}{q} = -\vec{E}(\vec{r})\dd\vec{r}$.

\textbf{Finite potential difference.} $\Delta V = V(\vec{r}) - V(\vec{r_0}) = \displaystyle\frac{\Delta U}{q} = 
-\displaystyle\int_{\vec{r_0}}^{\vec{r}}\vec{E}\ \dd\vec{r}.$

From this expression it's deduced that in a region of space in which the electrostatic field is zero, the potential is
constant.

\subsection{Motion of electric charges in an electrostatic field}
Combining Newton's law with the equations already seen we can study the notion of electric charges in a electrostatic field.

\textbf{Acceleration.} $\vec{a}\bydef\vec{F}/m = q\vec{E}/m.$
If $\vec{E}$ is constant, $\vec{a}$ is constant, giving a uniform accelerated motion.

\textbf{Equations of motion for constant acceleration.} $$v = v_0 + at,\quad x = x_0 + v_0t + \frac{1}{2}at^2.$$

\textbf{Conservation of energy law.} $E = K + U = \frac{1}{2}mv^2 + U$ with $U(\vec{r}) = qV(\vec{r})$ is constant.

\subsection{Electrical dipole}
\textbf{Dipole.} System of two equal charges with opposite sign separated a distance $\vec{L}$.

\textbf{Dipolar momemtum.} $\vec{p} = q\vec{L}$. It goes from the negative to the positive charge as $\vec{L}$. It's a 
chemical property that indicates how the electric charge is distributed since it's proportional to the distance and the
magnitude of the charge.

\textbf{Polarization.} Dipoles comforming matter are rearranged in presence of an external electric field, polarizing the
material.

\textbf{Electric field generated by a dipole.} $\vec{E}(\vec{r}) = \displaystyle\frac{1}{4\pi\epsilon_0}\displaystyle
\frac{\vnorm{\vec{p}}}{\vnorm{\vec{r}}^3}\uvec{r}$.

\textbf{Electrostatic potential energy of a dipole in an external uniform electric field.} $U(\vec{r}) = -\innerp{\vec{p}, 
\vec{E^*}(\vec{r})}$ where $\vec{E^*}$ denotes an external uniform electric field.

Note that this energy is minimum when $\vec{p}\parallel\vec{E^*}$. There's a momemtum that tends to orientate $\vec{p}$ parallel
to $\vec{E^*}$.

\subsection{Continuous distributions of charge}
\textbf{Linear charge density.} Amount of electric charge distributed by unit of length. $\lambda = Q/L$ (C/m).

\textbf{Superficial charge density.} Amount of electric charge distributed by unit of area. $\sigma = Q/A$ (C/m$^2$).

\textbf{Volumetric charge density.} Amount of electric charge distributed by unit of volume. $\rho = Q/V$ (C/m$^3$).

\textbf{Electric field created by a continuous distribution of (volumetric) charge.} $\vec{E}(\vec{r}) = \displaystyle
\frac{1}{4\pi\epsilon_0}\displaystyle\iiint_V\frac{\dd q}{\vnorm{\vec{r}}^2}\uvec{r}$, where $\vec{r}$ goes from 
$\dd q$ to $\vec{r}$ and $\dd q = \rho\dd V$.

\textbf{Electric field created by a continuous distribution of (superficial) charge.} $\vec{E}(\vec{r}) = \displaystyle
\frac{1}{4\pi\epsilon_0}\displaystyle\iint_S\frac{\sigma\dd S}{\vnorm{\vec{r}}^2}\uvec{r}$ where $\sigma = \dd q/\dd S$.

\textbf{Electric field created by a continuous distribution of (linear) charge.} $\vec{E}(\vec{r}) = \displaystyle
\frac{1}{4\pi\epsilon_0}\displaystyle\int_L\frac{\lambda\dd L}{\vnorm{\vec{r}}^2}\uvec{r}$.

\textbf{Electric field created by an infinite thread at a distance $R$.} $\vec{E}(\vec{r}) = \displaystyle
\frac{\lambda}{2\pi\epsilon_0\vnorm{\vec{r}}}$.

\textbf{Electric field in an axis point of a charged ring.} $\vec{E} = \frac{1}{4\pi\epsilon_0}\frac{qz}{(a^2 + z^2)^{3/2}}$.

\textbf{Electric field generated by an infinite plane.} $\vec{E} = \sigma / 2\epsilon_0$.

\textbf{Potential in a volume.} $V(\vec{r}) = \displaystyle\frac{1}{4\pi\epsilon_0}\underset{V}{\iiint}
\frac{\rho\ \dd V}{\vnorm{\vec{r}}}$.

\subsection{Flux of the electrostatic field. Gauss' law}
\textbf{Gauss' law.} $\Phi_E = \displaystyle\oint\vec{E}\ \dd \vec{S} = \displaystyle\frac{Q}{\epsilon_0}$.

\end{multicols*}

% -------------------------------------------------------
\newpage
\begin{center}
\vspace{-50mm}
\Large{\vspace{-15mm}\textbf{Table of constants}}
\vspace{-.4mm}

\footnotesize
\begin{longtable}{ll}
    \hline\noalign{\smallskip}
    \textbf{Constant} & \textbf{Value} \\
    \noalign{\smallskip}\hline\noalign{\smallskip}
    Electron charge & $e = 1'6\cdot 10^{-19}$ C \\
    Electrical permittivity in vacuum & $\epsilon_0 = 8'85\cdot 10^{-12} \textrm{ C}^2$/N$\textrm{m}^2$ \\
    Coulomb's constant & $K_e = \frac{1}{4\pi\epsilon_0} = 9\cdot 10^9$ Nm$^2$/C$^2$ \\
    \noalign{\smallskip}\hline\noalign{\smallskip}
\end{longtable}

\end{center}
\end{document}
