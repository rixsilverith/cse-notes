\chapter{Review of vector calculus}
\thispagestyle{noheaders}

This chapter includes a brief review of the different mathematical tools that we will be using during the 
following chapters. These mainly include linear algebra and vector calculus stuff such as inner and cross
products, partial derivatives, multiple integrals and vector fields.

\section{Vectors and reference systems}

\begin{defn}[Unitary vector]
    Given a vector $\vec{v}$, its unitary vector (vector whose norm is 1), denoted by $\uvec{v}$, is given by
    \begin{equation}
        \uvec{v}\bydef\frac{\vec{v}}{\vnorm{\vec{v}}}.
    \end{equation}
\end{defn}

\begin{defn}[Inner product]\label{def:inner-product}
    Given two vectors $\vec{x}$, $\vec{y}\in\Rtn$, their inner product, denoted by $\innerp{\vec{x}, \vec{y}}$, is
    defined as
    \begin{equation}
        \innerp{\vec{x}, \vec{y}}\bydef\sum_{i=1}^n x_iy_i \bydef \vnorm{\vec{x}}\vnorm{\vec{y}}\cos(\theta).
    \end{equation}
\end{defn}

From the geometrical expression of the \nref{def:inner-product}, the one with the cosine, the following results can
be derived.

\begin{prop}[Orthogonality]
    Let $\vec{x}$, $\vec{y}\in\Rtn$, then $\vec{x}\perp\vec{y}\iff\innerp{\vec{x}, \vec{y}} = 0$.
\end{prop}

\begin{prop}
    The angle $\theta$ between two vectors $\vec{x}$, $\vec{y}\in\Rtn$ is given by
    \begin{equation}
        \theta = \arccos(\frac{\innerp{\vec{x}, \vec{y}}}{\vnorm{\vec{x}}\vnorm{\vec{y}}}).
    \end{equation}
\end{prop}

\begin{defn}[Cross product]
    Let $\vec{x}$, $\vec{y}\in\R^3$. The cross product of both vectors is defined as
    \begin{equation}
        \vec{x}\times\vec{y}\bydef\begin{vmatrix}
            \ihat & \jhat & \khat \\
            x_x & x_y & x_z \\
            y_x & y_y & y_z
        \end{vmatrix} \bydef\vnorm{\vec{x}}\vnorm{\vec{y}}\sin(\theta).
    \end{equation}
\end{defn}

\begin{remark}
    The vector $\vec{x}\times\vec{y}$ is orthogonal to the plane given by vectors $\vec{x}$ and $\vec{y}$.
\end{remark}

\section{Gradient, flux and conservative fields}

In a region of space we have a vector field (respectively scalar) when there is defined a vectorial magnitude (respectively
scalar) for each point in that region as a function of the position.

\begin{defn}[Gradient]
    The gradient of a scalar function $f$ indicates the rate of changes of some magnitude in a given direction. In 
    cartesian coordinates is defined as
    \begin{equation}
        \vec{\nabla} f\bydef \frac{\partial f}{\partial x}\ihat + \frac{\partial f}{\partial y}\jhat + 
        \frac{\partial f}{\partial z}\khat.
    \end{equation}
\end{defn}

\begin{note}
    The gradient of a function is a vector whose direction is the one of maximum growth of the function in that point, and
    its magnitude is the slope of the function at that point.
\end{note}

\begin{defn}[Flux of a vector field through a surface]
    \begin{equation}
        \int\vec{F}\dd{\vec{a}} = \int\vec{F}\cdot\uvec{n}\dd{a}.
    \end{equation}
\end{defn}

Intuitively, the flux of a vector field through a surface is defined as the number of force lines that goes through the 
surface. In vector calculus, surfaces are defined by the value of its area and they are assigned a unit vector $\uvec{n}$
perpendicular to it.

\begin{remark}
    Note that the flux of a vector field is a scalar magnitude.
\end{remark}

\begin{defn}[Conservative field]
    A vector field is said to be conservative if its vector function $\vec{E}(x, y, z)$ can be obtained as the gradient
    of an scalar magnitude. 
    \begin{equation}
        \vec{E}(x, y, z) = \grad V(x, y, z).
    \end{equation}
\end{defn}

If a particle moves from one position to another by the effect of a conservative force $\vec{F}$, the work done is 
independent of the followed path, depending uniquely in the initial and final points. Therefore, we can express the work
as the difference between the points of a physical magnitude which is a function of the position, known as \textbf{potential
energy} of the particle.
\begin{equation}
    \grad\vec{F} = -\Delta U = W\quad\quad\quad \grad\vec{E} = -\Delta V.
\end{equation}
