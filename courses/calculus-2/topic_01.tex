\chapter[Introduction to the space of several real variables]{Introduction to the space of \\ several real variables}
\thispagestyle{noheaders}

In this introductory chapter to the course we are extending elemental concepts of analysis such as the concept of function or 
convergence to the space of several real variables, namely $\Rtn$, which is defined as the $n$ times cartesian product of $\R$ 
by itself,
\begin{equation}
\Rtn\bydef\underbrace{\R\times\ldots\times\R}_{n\textrm{ times}} = \{\left(x_1, \ldots, x_n\right)\st x_i\in\R,\ \forall i\st
1 \leq i \leq n\}.
\end{equation}

\section{$\Rtn$ as a metric space}



\begin{defn}[Metric space]
A metric space is an ordered pair $\left(X, d\right)$ where $X$ is a set together with an application 
$\appl{d}{X\times X}{\R}$ known as \textit{metric}, such that for any $x, y, z\in X$ satisfies the properties
\begin{itemize}[itemsep = -2pt]
	\item $d(x, y) = 0\iff x = y$.
	\item $d(x, y) = d(y, x)$.
	\item $d(x, z) \leq d(x, y) + d(y, z)\quad$ (Triangle inequality).
\end{itemize}
\end{defn}

By the taking $z = x$ in the triangle inequality, together with the second and first properties, it is deduced that $d(x, y)\geq 0$
for any $x, y\in X$. The application $d$ is sometimes referred as \textit{distance function} or simply \textit{distance}.

\begin{example}
	$\R$ is a complete metric space with the distance function $d(x, y) = \abs{y - x}$, known as the \textit{euclidean distance}.
\end{example}

\begin{defn}[Normed vector space]
A normed vector space is a vector space $V$ together with an application $\appl{\vnorm{\cdot}}{V}{\R}$ that satisfies the 
properties
\begin{itemize}[itemsep = -2pt]
\item $\forall\vec{v}\in V\implies \vnorm{\vec{v}} \leq 0$; $\vnorm{\vec{v}} = 0\iff \vec{v} = \vec{0}$.
\item $\forall\vec{v}\in V, \forall\lambda\in\R\implies \vnorm{\lambda\vec{v}}\leq\abs{\lambda}\vnorm{\vec{v}}$.
\item $\forall\vec{v}, \vec{w}\in V \implies \vnorm{\vec{v} + \vec{w}} \leq \vnorm{\vec{v}} + \vnorm{\vec{w}}$.
\end{itemize}
\end{defn}

\begin{prop}
Every normed vector space is a metric space by defining the metric as $d(x, y) = \vnorm{y - x}$. Then, it's easy to check that
\begin{itemize}[itemsep = -2pt]
\item $\forall\vec{v}\in V\implies \vnorm{-\vec{v}} = \vnorm{\vec{v}}$.
\item $\forall\vec{v}, \vec{w}\in V\implies \abs{\vnorm{\vec{w}} - \vnorm{\vec{v}}} \leq \vnorm{\vec{w}} - \vnorm{\vec{v}}$.
\end{itemize}
If such space is complete, then it's said to be a Banach space.
\end{prop}

\section{$\Rtn$ as a normed vector space}

Since normed vector spaces are a special case of metric spaces, $\Rtn$ is a normed vector space when defining a norm on it. In 
order to do this we use the inner product. 

\begin{defn}[Inner product]\label{def:inner-product}
Let $\vec{x} = \left(x_1, \ldots, x_n\right)$, $\vec{y} = \left(y_1, \ldots, y_n\right)\in\Rtn$, the inner product in $\Rtn$
is defined as
\begin{equation}\label{eq:inner-product}
\innerp{\vec{x}, \vec{y}} \bydef \sum_{i=1}^n x_i y_i. \bydef\vnorm{\vec{x}}\vnorm{\vec{y}}\cos(\theta),
\end{equation}
where $\theta$ is the angle between vectors $\vec{x}$ and $\vec{y}$.
\end{defn}

\begin{defn}[Euclidean norm]\label{def:euclidean-norm}
	Given $\vec{x}\in\Rtn$, the euclidean norm of the vector $\vec{x}$ is defined as
	\begin{equation}
		\vnorm{x}\bydef\sqrt{\innerp{\vec{x}, \vec{x}}}\bydef\sqrt{\sum_{i=1}^n x_i^2}.
	\end{equation}
\end{defn}

%Then, the so called \textbf{euclidian norm} is defined as $\vnorm{\vec{x}} \bydef \sqrt{\innerp{\vec{x}, \vec{x}}} \bydef 
%\sqrt{\displaystyle\sum_{i=1}^n x_i^2}$.

Also, given the geometrical expression of the inner product \eqref{eq:inner-product}, $\innerp{\vec{x}, \vec{y}} = \vnorm{\vec{x}}
\vnorm{\vec{y}}\cos(\theta)$, since $\abs{\cos(\theta)}\leq 1$, we deduce the following important result.
\begin{lemma}[Cauchy-Schwartz inequality]
$\forall\vec{x}, \vec{y}\in\Rtn \implies \abs{\innerp{\vec{x}, \vec{y}}}\leq \vnorm{\vec{x}}\vnorm{\vec{y}}$.
\end{lemma}

%\begin{proof}
%TODO.
%\end{proof}

By making use of \ref{eq:inner-product} and \ref{def:euclidean-norm} we can now define the notion of angle between vectors.

\begin{defn}[Angle between vectors]
Given $\vec{v}$, $\vec{w}\in\Rtn$, the angle between the vectors $\vec{v}$ and $\vec{w}$ is the number $\theta\in[0, \pi]$ 
determined by the inner product as
\begin{equation}
\cos(\theta)_{\left(\vec{x}, \vec{y}\right)} = \frac{\innerp{\vec{x}, \vec{y}}}{\vnorm{\vec{x}}\vnorm{\vec{y}}}
\end{equation}
\end{defn}


% Orthogonal projection definition goes here

\section{Set topology in $\Rtn$}

Topology is in charge of describing the different subsets of $\Rtn$ depending on the place that its points occupy. The
following are basic concepts of set topology and can be generalize to any metric space without defining explicitly the metric.

\begin{defn}[Open ball]
Given $\vec{x_0}\in\Rtn$ and a real number $\epsilon > 0$, an open ball with centered in $\vec{x_0}$ with radius $\epsilon$ is the set
\begin{equation}
B_\epsilon\argopen(\vec{x_0}\argclose)\bydef\{\vec{x}\in\Rtn\st\vnorm{\vec{x} - \vec{x_0}} < \epsilon\}.
\end{equation}
\end{defn}

\begin{note}
	In 2-dimensional space, balls are often called \textit{disks}.
\end{note}

\begin{defn}[Pierced open ball]
Given $\vec{x_0}\in\Rtn$ and a real number $\epsilon > 0$, a pierced open ball centered in $\vec{x_0}$ with radius $\epsilon$ is the
set $B'_\epsilon\argopen(\vec{x_0}\argclose)\bydef B_\epsilon\argopen(\vec{x_0}\argclose) - \{\vec{x_0}\}$. In other words, it's an 
open ball excluding its centre.
\end{defn}

\begin{defn}[Open set]
	A subset $S\subset\Rtn$ is called \textit{open} $\iff\forall\vec{x_0}\in S$ there exists $\epsilon > 0$ depending on $\vec{x_0}$ such that $B_\epsilon\argopen(\vec{x_0}\argclose)\subset S$. In other words, if $\forall\vec{x_0}\in S$ there exists $\epsilon > 0$
	such that a point in $\Rtn$ belongs to $S$ as soon as its Euclidean distance from $\vec{x_0}$ is smaller than $\epsilon$. 
\end{defn}

\begin{note}
An open set can be seen as a set that contains a ball around each of its points or, equivalently, a set which doesn't contain any of
its boundary points.
\end{note}

\begin{prop}
The union of infinitely many open sets and the intersection of a finite number of open sets are open.
\end{prop}

\begin{defn}[Closed set]
	A subset $S\subset\Rtn$ is called \textit{closed}, denoted by $\conj{S}$, if its complementary (relative to the space that it is defined on) $S^C \bydef\Rtn\setminus S$ is open. 
\end{defn}

\begin{prop}
The union of a finite number of closed sets and the intersection of infinitely many closed sets are closed.
\end{prop}

\begin{prop}
	The full space $\Rtn$ and null set $\O$ are both open and closed sets, known as \textit{clopen sets}.
\end{prop}

Open and closed sets generalize the idea of an open and closed interval in the real line to higher dimensions.

\begin{defn}[Interior]
	The interior of a subset $S\subset\Rtn$, denoted by $\interior{S}$, is the largest open subset of $\Rtn$ contained in $S$. 
	A point that is in the interior of $S$ is an \textbf{interior point}.
	\begin{equation}
		\interior{S}\bydef\{\vec{x_0}\in S\st\exists\epsilon > 0\implies B_\epsilon\argopen(\vec{x_0}\argclose)\subset S\}.
	\end{equation}
\end{defn}

\begin{defn}[Exterior]
	The exterior of a subset $S\subset\Rtn$, denoted by $\exterior{S}$, is the complementary set of the closure of $S$, defined as
	\begin{equation}
		\exterior{S}\bydef\{\vec{x_0}\in\Rtn\st\exists\epsilon > 0\implies B_\epsilon\argopen(\vec{x_0}\argclose)\subset
		\left(\Rtn\setminus S\right)\}.
	\end{equation}
	The points in this set are called \textbf{exterior points}.
\end{defn}

\begin{defn}[Closure]
	The closure of a subset $S\subset\Rtn$, denoted by $\closure{S}$, consists of all points in $S$ together with its boundary.
	Mathematically, 
	\begin{equation}
		\closure{S}\bydef\interior{S}\cup\frontier{S}.
	\end{equation}
	It is the smallest closed subset of $\Rtn$ contained in $S$, and its points are called \textbf{points of closure} of $S$.
\end{defn}

\begin{defn}[Boundary]
	The boundary, or frontier, of a subset $S\subset\Rtn$, denoted by $\frontier{S}$, is the set of points in the border of the
	set that can be approached both from $S$ and from the outside of $S$. More precisely, 
	\begin{equation}
		\frontier{S}\bydef\{\vec{x_0}\in\Rtn\st\exists\epsilon > 0\implies B_\epsilon
		\argopen(\vec{x_0}\argclose)\cap S\neq\O\land B_\epsilon\argopen(\vec{x_0}\argclose)\cap\left(\Rtn\setminus S\right)
		\neq\O\}.
	\end{equation}
	Equivalently, the boundary can be defined as $\frontier{S}\bydef\closure{S}\setminus\interior{S}$, and its points are called
	\textbf{boundary points}.
\end{defn}

\begin{prop}
	Let $S\subset\Rtn$, then the following properties hold:
	\begin{itemize}[itemsep = -2pt]
		\item $\interior{S}\cup\exterior{S}\cup\frontier{S} = \Rtn$.
		\item $\interior{S}\cap\exterior{S} = \interior{S}\cap\frontier{S} = \exterior{S}\cap\frontier{S} = \O$.
	\end{itemize}
\end{prop}

\begin{defn}[Isolated point]
A point $\vec{x_0}$ of a subset $S\subset\Rtn$ is said to be isolated $\iff\forall\epsilon > 0, \exists B_\epsilon\argopen(\vec{x_0}
	\argclose)\st B_\epsilon\argopen(\vec{x_0}\argclose)\cap S = \{\vec{x_0}\}$.
\end{defn}

\begin{defn}[Limit point]
Given $S\subset\Rtn$, a point $\vec{x_0}\in S$ is said to be a limit point, or accumulation point, of $S\iff\forall\epsilon > 0, 
	\exists B'_\epsilon\argopen(\vec{x_0}\argclose)\st B'_\epsilon\argopen(\vec{x_0}\argclose)\cap S\neq\O$. In other words, 
	if we can find points of $S$ as much close as we want to the point $\vec{x_0}$.
\end{defn}

\begin{defn}[Bounded set]
A set $S\subset\Rtn$ is said to be \textit{bounded} $\iff\exists M > 0\st\forall\vec{x}\in S\implies \vnorm{\vec{x}}\leq M$.
\end{defn}

\begin{defn}[Compact set]
A set $S\subset\Rtn$ is said to be \textit{compact} if it's closed and bounded.
\end{defn}

\begin{defn}[Convex set]
A set is said to be \textit{convex} if when tracing a segment between any two points we don't get out of the set.
\end{defn}

\section{Analytic geometry of lines and planes. Parametrization}
We'll now study some objects in the Euclidean plane and space. In particular, lines in $\R^2$ and planes in $\R^3$.

\section{Functions of the form $\Rtn\longrightarrow\R^m$}
So, depending on the values of $n$ and $m$ we can distinguise between two kinds of functions.

\begin{defn}[Scalar-valued function]\label{def:scalar-function}
	A scalar-valued function over (a subset of) $\Rtn$ is a mapping $\appl{f}{S\subseteq\Rtn}{\R}$, $\left(x_1, \ldots, x_n\right)
	\longmapsto f\left(x_1, \ldots, x_n\right)$.
\end{defn}

\begin{defn}[Vector-valued function]
	A vector-valued function over $\Rtn$ is a mapping $\appl{f}{S\subseteq\Rtn}{\R^m}$, $\vec{x}\longmapsto\vec{y} = f(\vec{x})$, where
	if $f$ goes from $\Rtn$ to $\R^{m>1}$ then
	\begin{equation}
		\vec{y} = f(\vec{x})\bydef\begin{bmatrix}y_1 \\ \vdots \\ y_m \end{bmatrix} = \begin{bmatrix}f_1(x_1, \ldots, x_n) \\
		\vdots \\ f_m(x_1, \ldots, x_n)\end{bmatrix}.
	\end{equation}
\end{defn}

\begin{remark}
	A vector function $\appl{f}{S\subseteq\Rtn}{\R^m}$ is given by $m$ scalar functions $\appl{f_i}{\Rtn}{\R}$ with $i = 1, \ldots,
	m$, where $f_i(x_1, \ldots, x_n)$ gives us the $i$-th component of $f$.
\end{remark}

\begin{example}
    The following function of several variables $f(x, y, z) = x + y + 2z$ is a scalar function.
\end{example}

% These are just the classical domain, image, etc., but extended to functions of several variables in $\Rtn$.

\begin{defn}[Domain]
Given a function $\stdvf$, the \textit{domain} of that function is the set of points in $\Rtn$ for which the function $f$ is 
defined, in other words, the set
\begin{equation}
\dom{f} \bydef \{\vec{x}\in\Rtn\st\exists f(\vec{x})\}\subseteq\Rtn.
\end{equation}
\end{defn}

\noindent For vector-valued functions; i.e. functions of the form $\stdvf$ with $m > 1$, the overall domain is the intersection of the
domains of each function defining $f$,
\begin{equation}
\dom{f}\bydef\bigcap_{i=1}^m\altdom{f_i}.
\end{equation}

\begin{defn}[Image]
Given a function $\stdvf$, the \textit{image} or \textit{range} of $f$ is the set
\begin{equation}
\image{f}\bydef\{\vec{y}\in\R^m\st \vec{y} = f\left(\vec{x}\right),\ \forall\vec{x}\in\altdom{f}\}\subseteq\R^m.
\end{equation}
\end{defn}

\begin{defn}[Graph]
Given a vector-valued function $\stdvf$, the graph of $f$ is the set
\begin{equation}
\fgraph{f}\bydef\{\left(\vec{x}, f\left(\vec{x}\right)\right)\st\vec{x}\in\altdom{f}\}\subset\R^{n + m}.
\end{equation}
\end{defn}

For $n\geq 3$ it's pretty difficult (actually, we can't) to draw graphs of functions. Nevertheless, this graphs
can be drawn using the concept of level sets.

\begin{defn}[Level sets]
Given a function $\gmvf$ and a scalar $c\in\R$, the level set of value $c$ for the function $f$ is a subset of the initial space
defined as
\begin{equation}
\levelset{c}{f} \bydef \{\vec{x}\in\altdom{f}\st f\left(\vec{x}\right) = c\}\subset\Rtn.
\end{equation}
\end{defn}

\begin{note}
    For $n=2$, this sets are called \textbf{level curves}, and for $n=3$, \textbf{level surfaces}.
\end{note}

Let us consider the graph of $\appl{f}{\R^2}{\R}$. Take its intersection with the plane $z = c$. This intersection gives 
us a curve in the initial space. This curve of level $c$ of $f$ is obtained by projecting the space curve onto the $XY$
plane.

\begin{example}
    Let $\appl{f}{\R^2}{\R}, (x, y)\longmapsto x^2 + y^2$. The level curve of level $c\geq 0$ of $f$ is a circle of radius
    $\sqrt{c}$ centered at the origin. The graph of $f$ is a surface called \textit{circular paraboloid}, which is obtained
    by rotating a parabola about its symmetry axis.
\end{example}

\begin{example}
    Let $\appl{f}{\R^2}{\R}, (x, y)\longmapsto x^2 - y^2$. When $c\neq 0$ we have a curve with equation $y = \pm\sqrt{x^2 - c}$.
    For $c > 0$, its level curve is a hyperbola in the region $\{(x, y)\st \abs{y} < x\}$. For $c < 0$, it is a hyperbola in
    $\{(x, y)\st\abs{y} > x\}$. Finally, when $c = 0$, we have the curve $y = \pm x$. In this case, the graph of $f$ is a 
    surface called \textit{hyperbolic paraboloid}.
\end{example}

\begin{example}
    Let $\appl{f}{\R^2}{\R}, (x, y)\longmapsto x^2 + 4y^2$. For each $c\geq 0$, the curve of level $c$ is an ellipse with
    equation $x^2 + 4y^2 = c$. Since $f > 0, \forall x, y\in\R$, for $c < 0$ the curve is the empty set. The graph of $f$ is
    a surface called \textit{elliptic paraboloid}.
\end{example}

\begin{remark}
    In functions of the form $f(x, y) = a_1x^2 + a_2y^2 + a_3$ with fixed coefficients $a_i\in\R$ and $a_1, a_2$ nonzero, the
    graphs can be classified as follows.
    \begin{itemize}[itemsep = -2pt]
        \item If $a_1, a_2$ have the same sign the graph of $f$ is an elliptic paraboloid. If $a_1 = a_2$, the graph is a 
            circular paraboloid.
        \item If $a_1, a_2$ have different sign the graph of $f$ is a hyperboloid paraboloid.
    \end{itemize}
\end{remark}

\section{$\epsilon-\delta$ definition of limit. Continuous functions}
\section{Topological properties of continuous functions}
