\chapter[Introduction to the space of several real variables]{Introduction to the space of \\ several real variables}
\thispagestyle{noheaders}

In this introductory chapter we are extending elemental concepts of analysis such as the concept of function or 
convergence to the space of several real variables, namely the $n$-dimensional Euclidean space $\Rtn$, which may be defined 
as the $n$ times cartesian product of $\R$ by itself or, equivalently, the set of $n$-tuples with real components,
\begin{equation}
\Rtn\bydef\underbrace{\R\times\ldots\times\R}_{n\textrm{ times}} \bydef \{\left(x_1, \ldots, x_n\right)\st x_i\in\R,\ \forall i\st
1 \leq i \leq n\}.
\end{equation}

\section{$\Rtn$ as an Euclidean normed vector space}

There are several ways of defining $\Rtn$, the most common one defines it as an $n$-dimensional Euclidean space. In these 
sections we'll give an equivalent, more general definition by using the concepts of metric space and normed vector space.

%definitions of $\Rtn$, the most common one defines it as an $n$-dimensional Euclidean space. In this 
%and the following section we give an alternative, more general definition of what $\Rtn$ is using the concepts of metric 
%spaces and normed vector spaces.

\begin{defn}[Metric space]\label{def:metric-space}
A metric space is an ordered pair $\left(X, d\right)$ where $X$ is a set together with an application 
$\appl{d}{X\times X}{\R}$, known as \textit{metric}, such that for any $x, y, z\in X$ satisfies the properties
\begin{itemize}[itemsep = -2pt]
	\item $d(x, y) = 0\iff x = y$.
	\item $d(x, y) = d(y, x)$.
	\item $d(x, z) \leq d(x, y) + d(y, z)\quad$ (Triangle inequality).
\end{itemize}
\end{defn}

\begin{remark}
By the taking $z = x$ in the triangle inequality, together with the second and first properties, it is deduced that $d(x, y)\geq 0$
for any $x, y\in X$. The application $d$ is sometimes referred to as \textit{distance function} or simply \textit{distance}.
\end{remark}

So, a metric space is just a set in which we can determine how close two elements belonging to it are.

\begin{defn}[Complete metric space]
    A metric space is said to be \textit{complete} $\iff$ all Cauchy sequences in it converge.
\end{defn}

\begin{example}
    $\R$ is a complete metric space by defining the distance function as $d(x, y) = \abs{y - x}$, known as 
    \textit{euclidean distance}. $\Q$ is also a metric space, but it's not a complete one.
\end{example}
\newpage
We know (and it can be easily but tediously proved) that the set of $n$-tuples $\Rtn$ conforms a vector space. Now, we need
to generalize the well known Euclidean distance in the real line to an $n$-dimensional Euclidean space, $E$. In order to do 
this we'll make use of the inner product.

\begin{defn}[Inner product]\label{def:inner-product}
    Let $E$ be an $n$-dimensional inner product space, and let $\vec{x} = (x_1, \ldots, x_n)$, $\vec{y} = (y_1, \ldots, y_n)
    \in V$. Then, the inner product between vectors $\vec{x}$ and $\vec{y}$ is defined as 
    $\innerp{\vec{x}, \vec{y}}\bydef\displaystyle\sum_{i=1}^n x_iy_i.$
\end{defn}

\noindent And using the \nref{def:inner-product} we can now define an application $\appl{\vnorm{\cdot}}{V}{F}$, which
we'll call \textit{Euclidean norm}, on our Euclidean vector space $E$ over the field $F$. 

\begin{defn}[Euclidean norm]\label{def:euclidean-norm}
    Let $E$ be an $n$-dimensional Euclidean vector space, and let $\vec{x}\in E$. The Euclidean norm of the vector $\vec{x}$
    is defined as $\vnorm{\vec{x}}\bydef\sqrt{\innerp{\vec{x}, \vec{x}}}$.
\end{defn}

\noindent Then, the \nref{def:euclidean-norm} will give us the magnitude of the vector $\vec{x}$. In other words, it'll
give us the distance between the initial and final points of the vector. Vector spaces with this application are known
as \textit{normed vector spaces}.

\begin{defn}[Normed vector space]\label{def:normed-vector-space}
    A normed vector space is a vector space $V$ over a field $F$ together with an application $\appl{\vnorm{\cdot}}{V}{F}$, 
    known as norm, that for all $\vec{v}, \vec{w}\in V$ satisfies the properties
\begin{itemize}[itemsep = -2pt]
\item $\vnorm{\vec{v}} \geq 0$; $\vnorm{\vec{v}} = 0\iff \vec{v} = \vec{0}$.
\item $\vnorm{\lambda\vec{v}}\leq\abs{\lambda}\vnorm{\vec{v}},\quad\forall\lambda\in F$.
\item $\vnorm{\vec{v} + \vec{w}} \leq \vnorm{\vec{v}} + \vnorm{\vec{w}}.\quad$(Triangle inequality).
\end{itemize}
\end{defn}

\begin{prop}
Every normed vector space is a metric space by defining the metric as $d(x, y) = \vnorm{y - x}$. 
\end{prop}

\begin{note}
    If a \nref{def:normed-vector-space} is complete, it's said to be a Banach space.
\end{note}

Since normed vector spaces are just a special case of metric spaces, $\Rtn$ can be defined as a \nref{def:metric-space}
whose metric is the \nref{def:euclidean-norm}, making it a \nref{def:normed-vector-space}.

\begin{note}
    Euclidean vector spaces are in fact normed vector spaces, which at the same time are inner product spaces.
\end{note}

\begin{prop}[Norm's properties]
    Let $V$ be a normed vector space. The following properties are true for all $\vec{v}, \vec{w}\in V$.
    \begin{align}
        \vnorm{-\vec{v}} &= \vnorm{\vec{v}}, \label{eq:norm-property-1} \\
        \abs{\vnorm{\vec{w}} - \vnorm{\vec{v}}} &\leq \vnorm{\vec{w}} - \vnorm{\vec{v}}. \label{eq:norm-property-2}
    \end{align}
\end{prop}

\begin{proof}
    The proof of \eqref{eq:norm-property-1} is trivial. To-do proof of \eqref{eq:norm-property-2}.
\end{proof}

\begin{prop}[Law of the parallelogram]
    Let $V$ be a normed vector space, and let $\vec{v}, \vec{w}\in V$. Then $\vnorm{\vec{v} + \vec{w}}^2 + \vnorm{\vec{v} 
    - \vec{w}}^2 = 2\vnorm{\vec{v}}^2 + 2\vnorm{\vec{w}}^2$.
\end{prop}

\begin{proof}
    This can be easily proved by developing the squared norms on the left side of the equality.
\end{proof}

\hide{
\begin{proof}
    We have, by definition, that 
    \begin{equation}
        \vnorm{\vec{v} + \vec{w}}^2 + \vnorm{\vec{v} - \vec{w}}^2 = \left(\vec{v} + \vec{w}\right)\left(
    \vec{v} + \vec{w}\right) + \left(\vec{v} - \vec{w}\right)\left(\vec{v} - \vec{w}\right).
    \end{equation}
    Developing these inner products using the distributive property we obtain 
    \begin{equation}
        \vnorm{\vec{v}}^2 + \vnorm{\vec{w}}^2 + 2\vec{v}\vec{w} + 
    \vnorm{\vec{v}}^2 + \vnorm{\vec{w}}^2 - 2\vec{v}\vec{w} = 2\vnorm{\vec{v}}^2 + 2\vnorm{\vec{w}}^2.
    \end{equation}
\end{proof}
}
\hide{
\begin{proof}
    We have, by definition, that $\vnorm{\vec{v} + \vec{w}}^2 + \vnorm{\vec{v} - \vec{w}}^2 = \left(\vec{v} + \vec{w}\right)\left(
    \vec{v} + \vec{w}\right) + \left(\vec{v} - \vec{w}\right)\left(\vec{v} - \vec{w}\right)$. Developing these inner 
    products using the distributive property we obtain $\vnorm{\vec{v}}^2 + \vnorm{\vec{w}}^2 + 2\vec{v}\vec{w} + 
    \vnorm{\vec{v}}^2 + \vnorm{\vec{w}}^2 - 2\vec{v}\vec{w} = 2\vnorm{\vec{v}}^2 + 2\vnorm{\vec{w}}^2$.
\end{proof}}

We now define some concepts in relation with a \nref{def:normed-vector-space}.

\begin{defn}[Unit vector]
    A vector $\vec{v}\in\Rtn$ is said to be a unit vector $\iff\vnorm{\vec{v}} = 1$.
\end{defn}

\begin{remark}
    Note that for any nonzero vector $\vec{v}$, $\uvec{v}\bydef\vec{v} / \vnorm{\vec{v}}$ is a unit vector.
\end{remark}

\begin{defn}[Orthogonality]
    A set of nonzero vectors $\{\vec{v_1}, \ldots, \vec{v_n}\}$ is an orthogonal set $\iff\forall i\neq j\implies
    \innerp{\vec{v_i}, \vec{v_j}} = 0$.
\end{defn}

\begin{defn}[Orthonormal set]
    A set of vectors $\set{\vec{v_1}, \ldots, \vec{v_n}}$ is an orthonormal set if it is an orthogonal set and for all 
    $\vec{v_i}\implies\vnorm{\vec{v_i}} = 1$.
\end{defn}

\begin{example}
    The set $\set{\ihat = (1, 0, 0),\ \jhat = (0, 1, 0),\ \khat = (0, 0, 1)}$ is an orthonormal set, known as the canonical 
    basis of $\R^3$.
\end{example}

Geometrically, using the \nref{def:euclidean-norm}, the \nref{def:inner-product} between two vectors $\vec{x}$, $\vec{y}\in
\Rtn$ can be defined as $\innerp{\vec{x}, \vec{y}}\bydef\vnorm{\vec{x}}\vnorm{\vec{y}}\cos(\theta)$, where $\theta$ is the 
angle between both vectors. This angle can be easily computed by solving for $\theta$.

\begin{defn}[Angle between vectors]\label{def:angle-vectors}
    Let $\vec{x}$, $\vec{y}\in\Rtn$. The angle $\theta$ between both vectors is defined as $\theta_{(\vec{x}, \vec{y})}
    \bydef\arccos(\frac{\innerp{\vec{x}, \vec{y}}}{\vnorm{\vec{x}}\vnorm{\vec{y}}}).$
\end{defn}

Using the notion of \nref{def:angle-vectors} we can define projections between them.

\begin{defn}[Orthogonal projection]\label{def:orthogonal-projection}
    The projection of some $\vec{v}$ over another vector $\vec{w}$ is the vector $\ovproj{\vec{w}}{\vec{v}}\bydef\left(
    \vnorm{\vec{v}}\cos(\theta)\right)
    \uvec{w}$, where $\theta$ is the angle between $\vec{v}$ and $\vec{w}$.
\end{defn}

\begin{remark}
    Note that this projection is the product of $\vec{w}$ by a scalar. In other words, the projection of $\vec{v}$ over
    $\vec{w}$ and $\vec{w}$ are on the same line.
\end{remark}

Using the relation between the angle and the inner product we can write the \nref{def:orthogonal-projection} of $\vec{v}$ 
over $\vec{w} $in terms of the \nref{def:inner-product}.
\begin{equation}
    \ovproj{\vec{w}}{\vec{v}}\bydef\left(\vnorm{\vec{v}}\cos(\theta)\right)\uvec{w} = \left(\frac{\vnorm{\vec{v}}}{\vnorm{\vec{w}}}
    \frac{\innerp{\vec{v}, \vec{w}}}{\vnorm{\vec{v}}\vnorm{\vec{w}}}\right)\vec{w} = \left(\frac{\innerp{\vec{v}, 
    \vec{w}}}{\vnorm{\vec{w}}^2}
\right)\vec{w}.
\end{equation}
\newpage
\begin{figure}[htbp]\label{fig:ortho-proj}
    \centering\incfig[.5\textwidth]{ortho-proj}
    \caption{Geometrically, on the plane $\R^2$, the projection of $\vec{v}$ over $\vec{w}$ is the vector whose final point is the intersection
between the line containing $\vec{w}$ and the line which is perpendicular to $\vec{w}$ containing the end point of $\vec{v}$.}
\end{figure}

Also, given the geometrical expression of the \nref{def:inner-product}, $\innerp{\vec{x}, \vec{y}} = \vnorm{\vec{x}}
\vnorm{\vec{y}}\cos(\theta)$, since $\abs{\cos(\theta)}\leq 1$, we deduce the following important result.

\begin{lemma}[Cauchy-Schwartz inequality]
    Let $\vec{x}, \vec{y}\in\Rtn$. Then, the inequality $\abs{\innerp{\vec{x}, \vec{y}}}\leq \vnorm{\vec{x}}\vnorm{\vec{y}}$
    holds.
\end{lemma}

%\begin{proof}
%TODO.
%\end{proof}

\hide{
By making use of \ref{eq:inner-product} and \ref{def:euclidean-norm} we can now define the notion of angle between vectors.

\begin{defn}[Angle between vectors]
Given $\vec{v}$, $\vec{w}\in\Rtn$, the angle between the vectors $\vec{v}$ and $\vec{w}$ is the number $\theta\in[0, \pi]$ 
determined by the inner product as
\begin{equation}
\cos(\theta)_{\left(\vec{x}, \vec{y}\right)} = \frac{\innerp{\vec{x}, \vec{y}}}{\vnorm{\vec{x}}\vnorm{\vec{y}}}
\end{equation}
\end{defn}
}

\section{Set topology in $\Rtn$}

Topology is in charge of describing the different subsets of $\Rtn$ depending on the place that its points occupy. The
following are basic concepts of set topology and can be generalize to any metric space without defining explicitly the metric.

\begin{defn}[Open ball]\label{def:open-ball}
Given $\vec{x_0}\in\Rtn$ and a real number $\epsilon > 0$, an open ball centered in $\vec{x_0}$ with radius $\epsilon$ is the set
\begin{equation}
B_\epsilon\argopen(\vec{x_0}\argclose)\bydef\{\vec{x}\in\Rtn\st\vnorm{\vec{x} - \vec{x_0}} < \epsilon\}.
\end{equation}
\end{defn}

\begin{remark}
    A closed ball is defined in the same way but considering $\vnorm{\vec{x} - \vec{x_0}}\leq\epsilon$, and it's denoted 
    by $\cball{\epsilon}{\vec{x_0}}$.
\end{remark}

\begin{note}
	In 2-dimensional space, balls are often called \textit{disks}.
\end{note}

\begin{defn}[Pierced open ball]
Given $\vec{x_0}\in\Rtn$ and a real number $\epsilon > 0$, a pierced open ball centered in $\vec{x_0}$ with radius $\epsilon$ is the
set $B'_\epsilon\argopen(\vec{x_0}\argclose)\bydef B_\epsilon\argopen(\vec{x_0}\argclose)\setminus\{\vec{x_0}\}$. In other words, it's an 
open ball excluding its centre.
\end{defn}

\begin{defn}[Open set]\label{def:open-set}
    A subset $S\subset\Rtn$ is an open set $\iff\forall\vec{x_0}\in S$ there exists $\epsilon > 0$ depending on $\vec{x_0}$ 
    such that $B_\epsilon\argopen(\vec{x_0}\argclose)\subset S$. In other words, if $\forall\vec{x_0}\in S$ there exists 
    $\epsilon > 0$ such that a point in $\Rtn$ belongs to $S$ as soon as its Euclidean distance from $\vec{x_0}$ is smaller 
    than $\epsilon$. 
\end{defn}

\begin{note}
An open set can be seen as a set that contains a ball around each of its points or, equivalently, a set which doesn't contain any of
its boundary points.
\end{note}

\begin{prop}
    Open sets have the following properties.
    \begin{itemize}[itemsep = -2pt, topsep = -1pt]
        \item Given a collection of open sets $\set{S_\alpha}_{\alpha\in I}\implies\displaystyle\bigcup_{\alpha\in I} S_\alpha$ is open.
        \item Let $S$ and $U$ be open sets $\implies S\cap U$ is open.
    \end{itemize}
%The union of infinitely many open sets and the intersection of a finite number of open sets are open.
\end{prop}

\begin{prop}
    Let $\vec{x_0}\in\Rtn$ and $\epsilon\in\R$. Then, $\forall\epsilon > 0$, $\oball{\epsilon}{\vec{x_0}}$ is open.
\end{prop}

\begin{proof}
    Let $\vec{x_1}\in\oball{\epsilon}{\vec{x_0}}$ and let $d = \vnorm{\vec{x_1} - \vec{x_0}}$. We need to find $\ell > 0$
    such that $\oball{\ell}{\vec{x_1}}\subset\oball{\epsilon}{\vec{x_0}}$ with $\ell = \epsilon - d$. Then, we have to
    prove that $\oball{\ell}{\vec{x_1}}\subset\oball{\epsilon}{\vec{x_0}}$. Let $\vec{y}\in\oball{\ell}{\vec{x_1}}$, we have,
    using the triangle inequality,
    that $\vnorm{\vec{y} - \vec{x_0}} = \vnorm{(\vec{y} - \vec{x_1}) + (\vec{x_1} - \vec{x_0})}\leq\vnorm{\vec{y} - 
    \vec{x_1}} + \vnorm{\vec{x_1} - \vec{x_0}} < \ell + d = \epsilon\iff\vec{y}\in\oball{\epsilon}{\vec{x_0}}$. 
\end{proof}

\begin{defn}[Closed set]\label{def:closed-set}
    A subset $S\subset\Rtn$ is a closed set, denoted by $\conj{S}$, if its complementary (relative to the space that 
    it is defined on) $S^C \bydef\Rtn\setminus S$ is open. 
\end{defn}

\begin{prop}
    Closed sets have the following properties.
    \begin{itemize}[itemsep = -2pt, topsep = -1pt]
        \item Given a collection of closed sets $\set{S_\alpha}_{\alpha\in I}\implies\displaystyle\bigcap_{\alpha\in I} 
            S_\alpha$ is closed.
        \item Let $S$ and $U$ be closed sets $\implies S\cup U$ is closed.
    \end{itemize}
%The union of a finite number of closed sets and the intersection of infinitely many closed sets are closed.
\end{prop}

\begin{prop}
    The full space $\Rtn$ and null set $\O$ are both open and closed sets, known as \textit{clopen sets}.
\end{prop}

Open and closed sets generalize the idea of an open and closed interval in the real line to higher dimensions.

\begin{defn}[Interior]
	The interior of a subset $S\subset\Rtn$, denoted by $\interior{S}$, is the largest open subset of $\Rtn$ contained in $S$. 
	\begin{equation}
		\interior{S}\bydef\{\vec{x_0}\in S\st\exists\epsilon > 0\implies B_\epsilon\argopen(\vec{x_0}\argclose)\subset S\}.
	\end{equation}
        A point that is in the interior of $S$ is an \textbf{interior point}.
\end{defn}

\begin{note}
    The interior of a set can be defined by the following property of maximality: if $U$ is open and $U\subset S$, then
    $U\subset\interior{S}$. In particular, if $S$ is open, then $\interior{S} = S$.
\end{note}

\begin{defn}[Exterior]
	The exterior of a subset $S\subset\Rtn$, denoted by $\exterior{S}$, is the complementary set of the closure of $S$, defined as
	\begin{equation}
		\exterior{S}\bydef\{\vec{x_0}\in\Rtn\st\exists\epsilon > 0\implies B_\epsilon\argopen(\vec{x_0}\argclose)\subset
		\left(\Rtn\setminus S\right)\}.
	\end{equation}
	The points in this set are called \textbf{exterior points}.
\end{defn}

\begin{defn}[Closure]
    The closure, or adherence, of a subset $S\subset\Rtn$, denoted by $\closure{S}$, consists of all points in $S$ 
    together with its boundary. Mathematically, 
    \begin{equation}
        \closure{S}\bydef\interior{S}\cup\frontier{S}.
    \end{equation}
    It is the smallest closed subset of $\Rtn$ contained in $S$, and its points are called \textbf{points of closure} of $S$.
\end{defn}

\begin{note}
    The closure of a set $S\subset\Rtn$, $\closure{S}$, can be defined by the following property of minimality: if $U$ is
    closed and $U\supset S$, then $U\supset\closure{S}$. In particular, if $S$ is closed then $\closure{S} = S$.
\end{note}

\begin{defn}[Boundary]
	The boundary, or frontier, of a subset $S\subset\Rtn$, denoted by $\frontier{S}$, is the set of points in the border 
        of the set that can be approached both from $S$ and from the outside of $S$. More precisely, 
	\begin{equation}
		\frontier{S}\bydef\{\vec{x_0}\in\Rtn\st\exists\epsilon > 0\implies B_\epsilon
		\argopen(\vec{x_0}\argclose)\cap S\neq\O\land B_\epsilon\argopen(\vec{x_0}\argclose)\cap\left(\Rtn\setminus S\right)
		\neq\O\}.
	\end{equation}
	Equivalently, the boundary can be defined as $\frontier{S}\bydef\closure{S}\setminus\interior{S}$, and its points are called
	\textbf{boundary points}.
\end{defn}

\begin{note}
    The boundary of a set $S\subset\Rtn$ can be also denoted by $\delta S$.
\end{note}

\begin{figure}[ht]
    \vskip.5cm
    \centering
    \incfig[.5\textwidth]{open-set}\vskip.3cm
    \caption{The set $S\subset\Rtn$ (in grey) is an open set, the black dashed line is its boundary, $\frontier{S}$, and
        their union is a closed set, the closure, $\closure{S}$.}
\end{figure}

\begin{prop}
	Let $S\subset\Rtn$, then the following properties hold.
        \begin{gather}
            \interior{S}\cup\exterior{S}\cup\frontier{S} = \Rtn, \\
            \interior{S}\cap\exterior{S} = \interior{S}\cap\frontier{S} = \exterior{S}\cap\frontier{S} = \O.
        \end{gather}
\end{prop}

\begin{defn}[Isolated point]\label{def:isolated-point}
A point $\vec{x_0}$ of a subset $S\subset\Rtn$ is said to be isolated $\iff\forall\epsilon > 0, \exists B_\epsilon\argopen(\vec{x_0}
	\argclose)\st B_\epsilon\argopen(\vec{x_0}\argclose)\cap S = \{\vec{x_0}\}$.
\end{defn}

\begin{defn}[Limit point]\label{def:limit-point}
Given $S\subset\Rtn$, a point $\vec{x_0}\in S$ is said to be a limit point, or accumulation point, of $S\iff\forall\epsilon > 0, 
	\exists B'_\epsilon\argopen(\vec{x_0}\argclose)\st B'_\epsilon\argopen(\vec{x_0}\argclose)\cap S\neq\O$. In other words, 
	if we can find points of $S$ as much close as we want to the point $\vec{x_0}$.
\end{defn}

\begin{defn}[Bounded set]\label{def:bounded-set}
A set $S\subset\Rtn$ is said to be \textit{bounded} $\iff\exists M > 0\st\forall\vec{x}\in S\implies \vnorm{\vec{x}}\leq M$.
\end{defn}

We can now define the important notion of \textit{compactness} in $\Rtn$.

\begin{defn}[Compact set]\label{def:compact-set}
A set $S\subset\Rtn$ is said to be \textit{compact} if it's closed and bounded.
\end{defn}

\begin{note}
    There's a more general definition of the notion of compacity which states that a set $S$ is compact if, for every
    collection of open sets whose union contains $S$, there exists a finite subcollection which also contains $S$.
    In $\Rtn$ the two definitions are equivalent (Heine-Borel theorem).
\end{note}



%\section{Analytic geometry of lines and planes. Parametrization}
%We'll now study some objects in the Euclidean plane and space. In particular, lines in $\R^2$ and planes in $\R^3$.

\section{Functions of the form $\Rtn\longrightarrow\R^m$}
Now that we have defined $\Rtn$, we are interested in studying functions of the form $\stdvf$. Depending on the values of 
$n$ and $m$ we can distinguise between two kinds of functions.

\begin{defn}[Scalar-valued function]\label{def:scalar-function}
	A scalar-valued function over (a subset of) $\Rtn$ is a mapping $\appl{f}{S\subseteq\Rtn}{\R}$, $\left(x_1, \ldots, x_n\right)
	\longmapsto f\left(x_1, \ldots, x_n\right)$.
\end{defn}

\begin{defn}[Vector-valued function]\label{def:vector-function}
	A vector-valued function over $\Rtn$ is a mapping $\appl{f}{S\subseteq\Rtn}{\R^m}$, $\vec{x}\longmapsto\vec{y} = f(\vec{x})$, where
	if $f$ goes from $\Rtn$ to $\R^{m>1}$ then
	\begin{equation}
		\vec{y} = f(\vec{x})\bydef\begin{bmatrix}y_1 \\ \vdots \\ y_m \end{bmatrix} = \begin{bmatrix}f_1(x_1, \ldots, x_n) \\
		\vdots \\ f_m(x_1, \ldots, x_n)\end{bmatrix}.
	\end{equation}
\end{defn}

\begin{remark}
	A vector function $\appl{f}{S\subseteq\Rtn}{\R^m}$ is given by $m$ scalar functions $\appl{f_i}{\Rtn}{\R}$ with $i = 1, \ldots,
	m$, where $f_i(x_1, \ldots, x_n)$ gives us the $i$-th component of $f$.
\end{remark}

\begin{example}
    The following function of several variables $f(x, y, z) = x + y + 2z$ is a scalar function.
\end{example}

% These are just the classical domain, image, etc., but extended to functions of several variables in $\Rtn$.

\begin{defn}[Domain]
Given a function $\stdvf$, the \textit{domain} of that function is the set of points in $\Rtn$ for which the function $f$ is 
defined, in other words, the set
\begin{equation}
\dom{f} \bydef \{\vec{x}\in\Rtn\st\exists f(\vec{x})\}\subseteq\Rtn.
\end{equation}
\end{defn}

\noindent For vector-valued functions; i.e. functions of the form $\stdvf$ with $m > 1$, the overall domain is the intersection of the
domains of each function defining $f$,
\begin{equation}
\dom{f}\bydef\bigcap_{i=1}^m\dom{f_i}.
\end{equation}

\begin{defn}[Image]
Given a function $\stdvf$, the \textit{image} or \textit{range} of $f$ is the set
\begin{equation}
\image{f}\bydef\{\vec{y}\in\R^m\st \vec{y} = f\left(\vec{x}\right),\ \forall\vec{x}\in\altdom{f}\}\subseteq\R^m.
\end{equation}
\end{defn}

\begin{defn}[Graph]
Given a vector-valued function $\stdvf$, the graph of $f$ is the set
\begin{equation}
\fgraph{f}\bydef\{\left(\vec{x}, f\left(\vec{x}\right)\right)\st\vec{x}\in\altdom{f}\}\subset\R^{n + m}.
\end{equation}
\end{defn}

For $n\geq 3$ it's pretty difficult (actually, we can't) to draw graphs of functions. Nevertheless, this graphs
can be drawn using the concept of level sets.

\begin{defn}[Level sets]
Given a function $\gmvf$ and a scalar $c\in\R$, the level set of value $c$ for the function $f$ is a subset of the initial space
defined as
\begin{equation}
\levelset{c}{f} \bydef \{\vec{x}\in\altdom{f}\st f\left(\vec{x}\right) = c\}\subset\Rtn.
\end{equation}
\end{defn}

\begin{note}
    For $n=2$, this sets are called \textbf{level curves}, and for $n=3$, \textbf{level surfaces}.
\end{note}

Let us consider the graph of $\appl{f}{\R^2}{\R}$. Take its intersection with the plane $z = c$. This intersection gives 
us a curve in the initial space. This curve of level $c$ of $f$ is obtained by projecting the space curve onto the $XY$
plane.

\begin{example}
    Let $\appl{f}{\R^2}{\R}, (x, y)\longmapsto x^2 + y^2$. The level curve of level $c\geq 0$ of $f$ is a circle of radius
    $\sqrt{c}$ centered at the origin. The graph of $f$ is a surface called \textit{circular paraboloid}, which is obtained
    by rotating a parabola about its symmetry axis.
\end{example}

\begin{example}
    Let $\appl{f}{\R^2}{\R}, (x, y)\longmapsto x^2 - y^2$. When $c\neq 0$ we have a curve with equation $y = \pm\sqrt{x^2 - c}$.
    For $c > 0$, its level curve is a hyperbola in the region $\{(x, y)\st \abs{y} < x\}$. For $c < 0$, it is a hyperbola in
    $\{(x, y)\st\abs{y} > x\}$. Finally, when $c = 0$, we have the curve $y = \pm x$. In this case, the graph of $f$ is a 
    surface called \textit{hyperbolic paraboloid}.
\end{example}

\begin{example}
    Let $\appl{f}{\R^2}{\R}, (x, y)\longmapsto x^2 + 4y^2$. For each $c\geq 0$, the curve of level $c$ is an ellipse with
    equation $x^2 + 4y^2 = c$. Since $f > 0, \forall x, y\in\R$, for $c < 0$ the curve is the empty set. The graph of $f$ is
    a surface called \textit{elliptic paraboloid}.
\end{example}

\begin{remark}
    In functions of the form $f(x, y) = a_1x^2 + a_2y^2 + a_3$ with fixed coefficients $a_i\in\R$ and $a_1, a_2$ nonzero, the
    graphs can be classified as follows.
    \begin{itemize}[itemsep = -2pt]
        \item If $a_1, a_2$ have the same sign the graph of $f$ is an elliptic paraboloid. If $a_1 = a_2$, the graph is a 
            circular paraboloid.
        \item If $a_1, a_2$ have different sign the graph of $f$ is a hyperboloid paraboloid.
    \end{itemize}
\end{remark}

\section{Limits in $\Rtn$. Convergence of sequences and continuous functions}

The notions of \nref{def:open-ball}, \nref{def:compact-set} and \nref{def:closed-set} previously seen are important 
to extend the concept of \textit{limit} of a sequence and of a function to $n$ dimensions.

\begin{defn}[Sequence in $\Rtn$]
    A sequence in $\Rtn$ is an ordered set of points $\set{\vec{x_i}\st i\in\N}\subset\Rtn$, usually denoted by 
    $\set{\vec{x_i}}_{i\in\N}$.
\end{defn}

\begin{defn}[Convergent sequence]
    A sequence $\set{\vec{x_i}}\subset\Rtn$ is convergent to $\vec{L}\in\Rtn$ if $\forall\delta > 0$ there exists $n_0\in\N$
    such that if $i > n_0$ then $\vec{x_i}\in\oball{\delta}{\vec{L}}$, where $\vec{L}$ is the limit of the sequence.
\end{defn}

\begin{note}
    This definition generalizes the notion of convergence seen for sequences in $\R$ by substituting the absolute value 
    over $\R$ by the Euclidean norm over $\Rtn$.
\end{note}

We now describe two important theorems.

\begin{theorem}[Characterization of closed sets by sequences]
    A set $S\subset\Rtn$ is closed $\iff$ for every sequence $\set{\vec{x_i}}$ contained in $S$ which converges, its limit
    is also in $S$.
\end{theorem}

\begin{defn}[Subsequence in $\Rtn$]\label{def:subsequence-rn}
    A subsequence $\set{\vec{y_m}}$ of a sequence $\set{\vec{x_i}}$ is one such that for every $m\in\N$ we have $\vec{y_m} =
    \vec{x_{i_m}}$, where the sequence of positive integers $i_m$ is increasing. 
\end{defn}

Using the notion of \nref{def:subsequence-rn} we extend the Bolzano-Weierstrass theorem in $\R$ to $n$ dimensions.

\begin{theorem}[Bolzano-Weierstrass theorem in $\Rtn$]
    Let $\set{\vec{x_i}}\subset\Rtn$ be a sequence contained in a \nref{def:bounded-set}. Then, there exists a subsequence
    of $\set{\vec{x_i}}$ which converges. In particular, if $\set{\vec{x_i}}$ is contained in a \nref{def:compact-set} $K$,
    then it has a convergent subsequence $\set{\vec{y_m}}$ to a limit $\vec{L}\in K$.
\end{theorem}

\begin{remark}
    Using the definition of convergence, we can see that a sequence $\set{\vec{x_i}}\subset\Rtn$ converges to $\vec{L}\iff$
    the sequence of positive real numbers $d(\vec{x_i}, \vec{L})$ converges to 0. \wtf
\end{remark}

\begin{defn}[Domain]\label{def:domain-set}
    A set $S\subset\Rtn$ is a domain if it's open and connected.
\end{defn}

To extend the notion of limit of a function from $\R$ to $n$ dimensions it's enough to replace the absolute value
by the \nref{def:euclidean-norm}.

\hide{
To extend the notion of limit of a function $\appl{f}{S\subset\Rtn}{\R^m}$ to $n$ dimensions when the variable of the 
function converges to a point let's recall the case of $\R$. In this case, 
\begin{equation}
    \lim_{x\to x_0} f(x) = L\iff \forall\epsilon > 0,\ \exists\delta(\epsilon, x_0) = \delta > 0 \st 0 < \abs{x - x_0} < \delta
        \implies\abs{f(x) - L} < \epsilon.
\end{equation}
The generalization of this concept of limit of a function from $\R$ to $\Rtn$ is done by replacing the absolute value
by the Euclidean norm.
}

\begin{defn}[Functional limit]\label{def:functional-limit}
    Let $S$ be an \nref{def:open-set} in $\Rtn$, let $\appl{f}{S}{\R^m}$ be a function and let $\vec{x_0}\in\dom{f}$ be
    a \nref{def:limit-point}. Then, $f$ has limit $\vec{L}$ in $\vec{x_0}$ when $\vec{x}\longrightarrow\vec{x_0}
    \iff\forall\epsilon > 0,\ \exists \delta(\epsilon) = \delta > 0 \st
    \forall\vec{x}\in S,\ 0 < \vnorm{\vec{x} - \vec{x_0}} < \delta\implies\vnorm{f(\vec{x}) - \vec{L}} < \epsilon$.
\end{defn}

\begin{remark}
    \Eref{def:functional-limit} can also be written in terms of open balls.
    \begin{equation}
        \lim_{\vec{x}\to\vec{x_0}}f(\vec{x})\iff\forall\oball{\epsilon}{\vec{L}},\ \exists\oball{\delta(\epsilon)}{\vec{x_0}}
            \st f\left(\oball{\delta(\epsilon)}{\vec{x_0}}\right)\subset\oball{\epsilon}{f(\vec{x_0})}.
    \end{equation}
\end{remark}

\begin{note}
    Why do we want $S$ to be open? This makes that, for $\delta > 0$, the condition $0 < \vnorm{\vec{x} - \vec{x_0}} < \delta$
    is satisfied for at least a point $\vec{x}\in S\setminus\set{\vec{x_0}}$, that is, that there exists points of $S$
    arbitrarily close to $\vec{x_0}$ but different from it. For instance, this allows that it can happen the case 
    $\lim_{\vec{x}\to\vec{x_0}}f(\vec{x}) = \vec{L}$ even when $f(\vec{x_0})\neq \vec{L}$.
\end{note}

\begin{note}
The concept of limit of functions over $\Rtn$, as in $\R$, can be defined similarly by using sequences. We have 
$\lim_{\vec{x}\to\vec{x_0}} f(\vec{x}) = \vec{L}\iff$ for every sequence $\set{\vec{x_i}}$ converging to $\vec{x_0}$, the
    sequence $f(\vec{x_i})$ converges to $\vec{L}$.

    In particular, this definition implies the following: Suppose that there exists the limit $\lim_{\vec{x}\to\vec{\x_0}}
    = \vec{L}$ and let $\set{\vec{x_i}}$, $\set{\vec{y_i}}$ be any two sequences converging to $\vec{x_0}$. If 
    $f(\vec{x_i})\longrightarrow\vec{c}$ and $f(\vec{y_i})\longrightarrow\vec{c'}$ when $i\longrightarrow\infty$, then
    $\vec{c} = \vec{c'} = \vec{L}$.

    This gives us a criteria to determine whether a certain function has a limit at a point. 
\end{note}

From the concept of \nref{def:functional-limit} in $\Rtn$ we can now define the concept of continuity of a function
at a point $\vec{x_0}$ of its domain.

\begin{defn}[Continuity in $\Rtn$]
    Let $S$ be an \nref{def:open-set} in $\Rtn$, let $\appl{f}{S}{\R^m}$ be a function and let $\vec{x_0}\in\dom{f}$. Then, 
    $f$ is continuous at $\vec{x_0}\iff f(\vec{x})\longrightarrow f(\vec{x_0})$ as $\vec{x}\longrightarrow\vec{x_0}$.
    %$\vec{x_0}\iff\lim_{\vec{x}\to\vec{x_0}} f(\vec{x}) = f(\vec{x_0})$. 
\end{defn}

\begin{remark}
    $f$ is continuous in $S\iff f$ is continuous at every point $\vec{x_0}\in S$.
\end{remark}

\begin{note}
    If $f$ is continuous at $\vec{x_0}$ then two scenarios can be going on. On one hand is that $\vec{x_0}$ is an 
    \nref{def:isolated-point} of $\dom{f}$ or it's a \nref{def:limit-point}, and in that case $f(\vec{x})\longrightarrow
    f(\vec{x_0})$ as $\vec{x}\longrightarrow\vec{x_0}$. On the other hand, if $f(\vec{x})\nlongrightarrow
    f(\vec{x_0})$ as $\vec{x}\longrightarrow\vec{x_0}$, $f$ is said to have an evitable discontinuity at $\vec{x} = \vec{x_0}$,
    which can be avoided by defining a new function given by
    \begin{equation}
        \tilde{f}(\vec{x})\bydef\begin{cases}
            f(\vec{x}),\quad\forall\vec{x}\neq\vec{x_0}, \\
            \lim_{\vec{x}\to\vec{x_0}} f(\vec{x}),\quad\forall\vec{x} = \vec{x_0}.
        \end{cases}
    \end{equation}
\end{note}

\begin{note}
    The properties of limits are inherited by continuous functions.
\end{note}

\begin{defn}[Limit along a curve]
    Let $f(\vec{x})$ be a scalar function of $n$ variables and let $\vec{x_0}\in\dom{f}$. Let $\Phi(t)$ be a continuous
    curve in $\Rtn$ of the form $\appl{\Phi}{\R}{\Rtn}$ such that $\Phi(0) = \vec{x_0}$ and $\exists\epsilon > 0$ such that
    $f(B'_{\epsilon}\argp{0} = (-\epsilon, 0)\cup(0, \epsilon))\subset\dom{f}$. Then, the limit of $f(\vec{x})$ along the 
    curve $\Phi(t)$ is
    \begin{equation}
        \lim_{t\to 0} f(\Phi(t)).
    \end{equation}
\end{defn}

\begin{note}
    For all $\vec{x_0}$ belonging to the boundary of the domain of $f$, $\boundary{\dom{f}}$, we have that the limit is only
    defined in a single direction of the curve. If along several curves the limit is not equal, then the limit doesn't exists.
\end{note}

\section{Properties of limits and continuous functions}

\begin{prop}[Uniqueness of the limit]
    The limit, if exists, is unique.
\end{prop}

The following properties about limits can be proved using the $\epsilon-\delta$ definition of limit.

\begin{prop}[Lineality of the limit]
    Given two functions $\appl{f, g}{\Rtn}{\R^m}$ with limits $\vec{L}$ and $\vec{M}$ respectively at 
    $\vec{x} = \vec{x_0}$, we have
    \begin{equation}
        \lim_{\vec{x}\to\vec{x_0}} \left(\lambda f(\vec{x})\pm \mu g(\vec{x})\right) = \lambda\vec{L}\pm \mu\vec{M},\quad
            \lambda, \mu\in\R
    \end{equation}
\end{prop}

\begin{remark}
    It can occur that the sum of two discontinuous functions at a certain $\vec{x_0}$ is continuous for that value.
    However, the sum of two continuous functions will be always continuous and the sum of a continuous function with a 
    non-continuous one will be always discontinuous.
\end{remark}

\begin{prop}
    Let $\appl{f, g}{\Rtn}{\R}$ with limits $\vec{L}$ and $\vec{M}$ respectively at $\vec{x} = \vec{x_0}$, then
    %\begin{equation}
        $\lim_{\vec{x}\to\vec{x_0}}\left(f(\vec{x})g(\vec{x})\right) = \vec{L}\cdot\vec{M}$.
    %\end{equation}
\end{prop}

\begin{remark}
    The produt of continuous functions is always continuous. Somehow, it can occur that the product of a continuous 
    function with a discontinuous one as well as the product of two discontinuous functions can be continuous.
\end{remark}

\begin{prop}
    Given $\appl{f}{\Rtn}{\R}$ such that $f(\vec{x}) = \vec{y}\neq\vec{0}$, we have that
    \begin{equation}
        \exists\lim_{\vec{x}\to\vec{x_0}}\frac{1}{f(\vec{x})} = \frac{1}{\vec{y}}.
    \end{equation}
    On the other hand, given $f, g$ continuous at $\vec{x_0}$ such that $g(\vec{x_0})\neq\vec{0}$ we have
    \begin{equation}
        \exists\lim_{\vec{x}\to\vec{x_0}}\left(h(\vec{x}) = \frac{f(\vec{x})}{g(\vec{x})}\right),
    \end{equation}
    therefore $h(\vec{x})$ is continuous at $\vec{x} = \vec{x_0}$.
\end{prop}

\begin{prop}
    Given a vector-valued function with $m$ scalar components $\appl{f}{\Rtn}{\R^m}$ we have that
    \begin{equation}
        \lim_{\vec{x}\to\vec{x_0}}f(\vec{x}) = \vec{L}\iff\lim_{\vec{x}\to\vec{x_0}} f_i(\vec{x}) = \vec{L_i},\quad\forall
            1 \leq i \leq m.
    \end{equation}
    In other words, a vector function is continuous $\iff$ each of its component functions $f_i$ is continuous.
\end{prop}

\begin{prop}
    Given $\appl{f}{\Rtn}{\R}$, if $f$ is continuous at $\vec{x} = \vec{x_0}$, $f(\vec{x}) > 0$ $(< 0)$ and $\vec{x_0}$ is
    a \nref{def:limit-point} of $\dom{f}$, then
    \begin{equation}
        \exists\epsilon > 0\st f(\vec{x}) > 0\ (< 0),\ \forall\vec{x}\in\oball{\epsilon}{\vec{x_0}}\cap\dom{f}.
    \end{equation}
\end{prop}

\begin{lemma}[Sandwich lemma]
    Given three functions $\appl{f, g, h}{\Rtn}{\R}$, if $\exists\epsilon > 0$ such that $f(\vec{x})\leq h(\vec{x})\leq 
    g(\vec{}x)$ for all $\vec{x}\in\oball{\epsilon}{\vec{x_0}}$, and moreover 
    \begin{equation}
        \lim_{\vec{x}\to\vec{x_0}}f(\vec{x}) = \lim_{\vec{x}\to\vec{x_0}}g(\vec{x}) = \vec{L}\implies
        \lim_{\vec{x}\to\vec{x_0}}h(\vec{x}) = \vec{L}. 
    \end{equation}
\end{lemma}

\begin{prop}
    Given $\appl{f, g}{\Rtn}{\R}$, if $f(\vec{x})\longrightarrow\vec{0}$ as $\vec{x}\longrightarrow\vec{x_0}$ and there
    exist two real numbers $M, \epsilon > 0$ such that $\abs{g(\vec{x})}\leq M,\ \forall\vec{x}\in
    \oball{\epsilon}{\vec{x_0}}\cap\dom{f}$, then
    \begin{equation}
        \exists\lim_{\vec{x}\to\vec{x_0}}\left(f(\vec{x})g(\vec{x})\right) = 0.
    \end{equation}
    In other words, the limit of a zero function at $\vec{x} = \vec{x_0}$ by a bounded function is zero.
\end{prop}

\begin{prop}
    Given $\appl{f}{\Rtn}{\R^p}$ and $\appl{g}{\R^p}{\R^m}$, $g\circ f$ is defined as $g\left(f(\vec{x})\right)$. Suppose
    that
    \begin{itemize}[itemsep = -2pt, topsep = -2pt]
        \item $\vec{x_0}$ is a \nref{def:limit-point} of $\dom{f}$ and $f(\vec{x})$ is continuous at $\vec{x_0}$,
        \item $f(\vec{x_0})$ is a \nref{def:limit-point} of $\dom{g}$ and $g(\vec{x})$ is continuous at $\vec{x_0}$,
        \item $\vec{x_0}$ is a \nref{def:limit-point} of $\dom{g\circ f}$ and $g\circ f$ is continuous at $\vec{x} = \vec{x_0}$,
    \end{itemize}
    then we have that
    \begin{equation}
        \lim_{\vec{x}\to\vec{x_0}}g\left(f(\vec{x})\right) = g\left(f(\vec{x_0})\right).
    \end{equation}
    The resulting function of the composition of two continuous functions is continuous in general.
\end{prop}

\hide{
\begin{prop}
    Let $\appl{f, g}{S\subset\Rtn}{\R^m}$ and let $\lim_{\vec{x}\to\vec{x_0}} f(\vec{x}) = \vec{L}$, 
    $\lim_{\vec{x}\to\vec{x_0}} g(\vec{x}) = \vec{M}$. Then, the following properties regarding the limit of a function hold.

    \begin{itemize}[itemsep = -2pt, topsep = -2pt]
        \item $\lim_{\vec{x}\to\vec{x_0}} \left(\lambda f(\vec{x})\pm \mu g(\vec{x})\right) = \lambda\vec{L}\pm \mu\vec{M},\
            \lambda, \mu\in\R$.
        \item $\lim_{\vec{x}\to\vec{x_0}}\left(fg\right)(\vec{x}) = \vec{LM},\quad\textrm{when }m = 1$.
        \item $\lim_{\vec{x}\to\vec{x_0}}\frac{1}{f(\vec{x})} = \frac{1}{\vec{L}},\quad m = 1, \vec{L}\neq\vec{0}$.
    \end{itemize}
\end{prop}

\begin{prop}[Composition of limits]
    Let $\appl{f}{S\subset\Rtn}{\R^m}$ and $\appl{g}{f(S)\subset\R^m}{\R^s}$ be two functions. Then,
    \begin{equation}
        \lim_{\vec{x}\to\vec{x_0}} f(\vec{x}) = \vec{L}\land\lim_{\vec{y}\to\vec{L}} g(\vec{y}) = \vec{M}\implies
        \lim_{\vec{x}\to\vec{x_0}} g\left(f(\vec{x})\right) = \vec{M}.
    \end{equation}
\end{prop}
}

\section{Topological properties of continuous functions}

\begin{defn}[Connected set]\label{def:connected-set}
    An \nref{def:open-set} $S\subset\Rtn$ is said to be connected $\iff\forall\vec{x}, \vec{y}\in S$ there exists a 
        continuous arc $\appl{\varphi}{[\vec{a}, \vec{b}]}{S\subset\Rtn}\st\varphi(\vec{a}) = \vec{x}\land\varphi(\vec{b})
        = \vec{y}$ and $\forall\vec{c}\in [\vec{a}, \vec{b}]\implies\varphi(\vec{c})\in S$. In other words, if we can trace
        a curve $\varphi$ from $\vec{x}$ to $\vec{y}$ without getting out of $S$.
\end{defn}

\begin{defn}[Homeomorphism]
    A function $\appl{\vec{f}}{\Rtn}{\R^m}$ is a homeomorphism if it is a continuous bijection and if 
    $\appl{\vec{\invers{f}}}{\R^m}{\Rtn}$ is also continuous.
\end{defn}

\begin{remark}
    Homeomorphisms preserve the topological properties of a set and they are sometimes called \textit{bicontinuous functions}.
\end{remark}

\begin{theorem}[Intermediate values theorem]
    Given a continuous function $\appl{f}{\Rtn}{\R}$ consider an open \nref{def:connected-set} $S\subset\dom{f}$ and a
    pair of points $\vec{x}, \vec{y}\in S$ such that
    \begin{equation}
        \begin{rcases}
            f(\vec{x}) = \vec{\alpha} \\
            f(\vec{y}) = \vec{\beta}
        \end{rcases}\forall\vec{\alpha} < \vec{\beta}\implies\forall\vec{\gamma}\st\vec{\alpha}\leq\vec{\gamma}\leq\vec{\beta},\ \exists\vec{z}\in S\st 
    f(\vec{z}) = \vec{\gamma}.
    \end{equation}
\end{theorem}

\begin{theorem}
    The image of an open \nref{def:connected-set} by a continuous function is an interval in $\R$.
\end{theorem}

\begin{theorem}
    Given a continuous function $\appl{f}{\Rtn}{\R^m}$ and a \nref{def:compact-set} $S\subset\dom{f}\implies f(S)\subset\R^m$
    is compact.
\end{theorem}

\begin{theorem}
    If $S\subset\Rtn$ is compact and the function $\appl{f}{S\subset\Rtn}{\R}$ is continuous in $S$, then $f$ is bounded
    in $S$ and $M = \sup_{\vec{x}\in S} f(S)$, $m = \inf_{\vec{x}\in S} f(S)$ with $M, m\in S$.
\end{theorem}

\begin{theorem}
    If $\appl{f}{\Rtn}{\R^m}$ is continuous and bijective from $S$ to $f(S)$ and $S$ is compact, then 
    $\appl{\invers{f}}{f(S)}{S}$ is also continuous.
\end{theorem}

\begin{note}
    Continuous and bijective functions such that its inverse is also continuous are called \textbf{homeomorphisms} and they
    preserve the topological properties of a set.
\end{note}

\begin{theorem}[Topological characterization of continuity]
    A function $\appl{f}{S\subset\Rtn}{\R^m}$ is continuous $\iff$ the following conditions are satisfied.
        \begin{itemize}[itemsep = -2pt, topsep = -2pt]
            \item If $U\subset f(S)$ is an open set $\implies$ the preimage $\invers{f}(U)$ is open.
            \item If $U\subset f(S)$ is a closed set $\implies$ the preimage $\invers{f}(U)$ is closed.        
        \end{itemize}
\end{theorem}

