\chapter[Elemental logic and demonstration methods]{Elemental logic and \\ demostration methods}
\thispagestyle{noheaders}

This first chapter is an introduction to classical logic, which will help us to understand how a mathematical
deduction is developed. We'll begin by studying some theory about \textit{propositional logic} and 
\textit{first order logic}, or \textit{predicate logic}, and then we'll move to the different demonstration methods 
used in mathematics.

\section{Propositional logic}

\begin{defn}[Logic proposition]
    Expression or sentence that can be either true or false. 
\end{defn}

\begin{note}
    In propositional logic there exists only two truth values: $T$ (true) and $F$ (false).
\end{note}

\begin{defn}[Simple proposition]
It makes reference to a unique content of truth. That is, the proposition is true or false by itself.
\end{defn}

\begin{defn}[Molecular proposition]
    A proposition composed of several simple ones, which are linked by logic connectors.
\end{defn}

\hide{
\noindent A proposition is said to be \textbf{simple} if it
    makes reference to an unique content of truth (that is, if it's true or false by itself) and it's represented
    by a capital letter; and \textbf{molecular} if it's composed of several simple propositions linked by logic
    connectors.
}

Now let's introduce the different logic connectors that are used to link different simple propositions.
\begin{itemize}[itemsep = -1pt, topsep = -1pt]
    \item\textbf{Negation ($\neg$)} It indicates the opposite truth value of the proposition next to it.
    \item\textbf{Conjunction ($\land$)} It indicates that both propositions are true.
    \item\textbf{Disjunction ($\lor$)} It indicates that one proposition is true, or the other, or both.
    \item\textbf{Conditional ($\implies$)} When the first proposition holds, the second is true. In other words, the 
        second proposition is a logic consequence of the first.
    \item\textbf{Logic equivalence ($\iff$)} Every time one of the two propositions is true, the other is also true, and 
        viceversa. $\left(P\implies Q\right)\land\left(Q\implies P\right)$.
\end{itemize}

\hide{
\begin{defn}[Negation]
    It indicates the opposite truth value of a proposition. Denoted by $\neg$.
\end{defn}

\begin{defn}[Conjunction]
    It indicates that both propositions are true at the same time. Denoted by $\land$.
\end{defn}

\begin{defn}[Disjunction]
    It indicates that one proposition is true, or the other, or both. Denoted by $\lor$.
\end{defn}

\begin{defn}[Conditional / Implication]
    When the first proposition holds, the second is true. In other words, the second proposition is
    a logic consequence of the first. Denoted by $\implies$.
\end{defn}

\begin{defn}[Logic equivalence]
    Every time one of the two propositions is true, the other is also true, and viceversa. Denoted by $\iff$.
\end{defn}
}

To represent the truth value of a complex proposition we use truth tables.

\begin{defn}[Truth table]
    A truth table is the representation of the truth or falsehood of a proposition from the truth value of
    all its simple elements.
\end{defn}

\noindent Truth tables are a useful tool to prove logic propositions.

\begin{example}
    The truth table corresponding to the above logic connectors is the following.
\end{example}
\begin{longtable}[c]{c|c|c|c|c|c|c}
    $P$ & $Q$ & $\neg P$ & $P\land Q$ & $P\lor Q$ & $P\implies Q$ & $P\iff Q$ \\ \hline
    0 & 0 & 1 & 0 & 0 & 1 & 1 \\
    0 & 1 & 1 & 1 & 0 & 1 & 0 \\
    1 & 0 & 0 & 1 & 0 & 0 & 0 \\
    1 & 1 & 0 & 1 & 1 & 1 & 1 \\
\end{longtable}

\begin{defn}[Tautology]\label{def:tautology}
    Proposition that is always true for any truth value. 
\end{defn}

\begin{example}
$P\lor\left(\neg P\right)$.
\end{example}

\begin{example}
    (Modus Ponens) $[P\land \left(P\implies Q\right)]\implies Q$.
\end{example}

\begin{defn}[Contradiction]\label{def:contradiction}
    Proposition that is always false for any truth value. 
\end{defn}

\begin{example}
$P\land\left(\neg P\right)$.
\end{example}

\begin{defn}[Equivalent propositions]
    Two logic propositions $P$ and $Q$ are equivalent, denoted $P\equiv Q$, if they have the same truth values 
    for the same truth values of all their simple elements. 
\end{defn}

\begin{prop}[De Morgan's laws]\label{prop:de-morgan-law-logic}
    Let $P$ and $Q$ be two simple propositions. Then, the following equivalences hold.
    \begin{equation}
        \neg\left(P\land Q\right)\equiv\left(\neg P\right)\lor\left(\neg Q\right),\quad\quad\neg\left(P\lor Q\right)
        \equiv\left(\neg P\right)\land\left(\neg Q\right).
    \end{equation}
\end{prop}

\begin{prop}[Distributive law]\label{prop:distributive-law}
    Let $P$, $Q$ and $R$ be three simple propositions. Then, the following equivalences hold.
    \begin{align}
        P\land\left(Q\lor R\right)&\equiv\left(P\land Q\right)\lor\left(P\land R\right),\\ P\lor\left(Q\land R\right)
                                  &\equiv\left(P\lor Q\right)\land\left(P\lor R\right).
    \end{align}
\end{prop}

\begin{proof}
    To prove \fulleref{prop:distributive-law} and \fulleref{prop:de-morgan-law-logic} it's enough to use truth tables.
\end{proof}

\begin{prop}[Sheffer conector]
    $P\mid Q\equiv\neg\left(P\land Q\right)$.
\end{prop}

\hide{
\subsection{Necessary and sufficient conditions}
Suppose that $P$ and $Q$ are simple logical propositions through this section.




\begin{example}
    Every continuous function is differentiable at some point.
\end{example}

\begin{example}
    Let $P_1$ be the proposition \textit{all dogs are mamals}, $P_2$ be \textit{all huskies are dogs} and $C$ be 
    \textit{all huskies are mamals}. The logic reasoning goes as follows. If $p = $\textit{dog}, $q =$\textit{mamal} and
    $r =$\textit{husky}, then $p\implies q$ (read as $p$ implies $q$) and $r\implies p$. Therefore, $r\implies q$.
\end{example}
}

\section{First-order (predicate) logic}

First-order logic, also known as predicate logic, is about the truth or falsehood of propositions that refer to the 
elements of a set, called \textbf{universal set}, $\uset$. 

In general, a set is a collection of elements. However, When an element $x$ is in a set $\uset$, it's said that the element 
belongs to the set, $x\in\uset$. Otherwise, it's said that the element doesn't belong to the set, $x\not\in\uset$.

\begin{example}
    Suppose that our universe is the real numbers, $\uset = \R$ and a proposition $P\equiv x^2 - 3x + 2 = 0$.
    This proposition will be true only when $x = 1$ or $x = 2$, and false otherwise, but the thing is that it
    has no truth value by itself.
\end{example}

\begin{defn}[Quantifiers]
    In predicate logic, quantifiers refer to a certain property of an element in a collection.
    \begin{itemize}[itemsep = -2pt, topsep = -1pt]
        \item\textbf{Existence quantifier, $\exists$.} States the existance of, at least, one element $x$ in 
            the set $\uset$ such that it makes true a proposition $P(x)$.
        \item\textbf{Universal quantifier, $\forall$.} States that for every element $x$ in the set $\uset$ a
            proposition $P(x)$ is true.
    \end{itemize}
\end{defn}

\begin{remark}
    The existence quantifier $\exists$ ensures the existence of at least one element $x$ that makes $P(x)$ true. 
    However if we want to denote that there exists a unique element $x$ that makes $P(x)$ true we use the
    $\uexists$ quantifier.
\end{remark}

\begin{note}
    Propositions involving quantifiers are true or false by itself.
\end{note}

If we have an expression with an existence quantifier, $\exists x$, $P(x)$, its negation is $\neg\left(
\exists x, P(x)\right)\equiv\forall x, \neg P(x)$. In the same way, if we have an expression with an universal
quantifier, $\forall x$, $P(x)$, its negation is given by $\neg\left(\forall x, P(x)\right)\equiv\exists x,
\neg P(x)$.

\section{Demonstration methods}
A mathematical demonstration, or proof, is a logical reasoning made from hypothesis until reaching a conclusion,
or thesis. These reasonings are based on axioms, which are premises considered as obvious, or in theorems
that have already been proved.

%\subsection{Direct proof}
%This method is based in reaching the conclusion, or thesis, from the hypothesis using deductions.

\begin{defn}[Direct proof]
    This method is based in reaching the conclusion, or thesis, from the starting hypothesis using deduction.
\end{defn}

\begin{example}
    Let $n$, $m\in\N$ be two powers of 3. We want to prove that $n + m$ never can be a power of $3$. To prove this we need
    to see what happens when $n = m$ and when $n\neq m$.
    \begin{itemize}[itemsep = -2pt, topsep = -1pt]
        \item Suppose $n = m$, then there exists $k\in\N$ such that $n = m = 3^k$. Thus, $n + m = 3^k + 3^k = 2\cdot 3^k$,
            which is not a power of three.
        \item Now suppose $n\neq m$ and $m < n$. Then there exists $k$, $\ell\in\N$ with $k < \ell$ such that $m = 3^k$ and
            $n = 3^\ell$. Thus, $n + m = 3^k + 3^\ell = 3^k\left(1 + 3^{\ell - k}\right)$. For this sum to be a power of 3,
            we need that $1 + 3^{\ell - k} = 3^p$ for some $p\in\N$. But $3^p$ is an odd number and $3^{\ell - k} + 1$ is even,
            therefore, the previous equation can't be true.
    \end{itemize}
\end{example}

\begin{defn}[Counterreciprocal]
    This method is based on the equivalence $P\implies Q\equiv \neg Q\implies\neg P$.
\end{defn}

\begin{remark}
    Note that $\left(P\implies Q\right)\iff \left(\neg Q\implies\neg P\right)$ is a \nref{def:tautology}.
\end{remark}

\begin{example}
    Let $p\in\N$ be a prime number. We'll prove that if $p > 2$, then $p$ is an odd number. For this, let's call 
    $P\equiv\left(p\textrm{ is prime}\right)\land\left(p > 2\right)$ and $Q\equiv p\textrm{ is odd}$. 
    
    If we want to prove
    that $P\implies Q$, using the counterreciprocal we'll prove that $\neg Q\implies\neg P$. In other words, if $p$ is even,
    then $p\leq 2$ or $p$ is not prime.

    Let $p$ be an even number; i.e. $p = 2k$ for some $k\in\N$. If $k = 2$, then $p = 2$, and if $k > 1$, then $p$ is not 
    prime since it's divisible by 2. Therefore, if $p > 2$ and $p$ is prime, then $p$ is odd.
\end{example}

\begin{defn}[Proof by contradiction]
    The proof by contradiction method consists on supposing that the proposition or implication we want to prove is false,
    leading us to a contradiction, which means that are supposition is wrong.
\end{defn}

\begin{example}
    Let's prove that there not exists any natural number greater than the rest of them. Thus, let's suppose that it exists
    $n\in\N$ which is the greatest natural number, but $n + 1$ is also a natural number and $n + 1 > n$, which is a 
    contradiction.
\end{example}

\begin{note}
    A proof by contradiction and by counterreciprocal are known as indirect proofs.
\end{note}

\begin{defn}[Mathematical induction]
    Induction is used to prove that a certain proposition $P$ is true for any natural number $n$. It consists on two steps.
    \begin{itemize}[itemsep = -2pt, topsep = -1pt]
        \item\textbf{Initial case.} Here we prove that our proposition is true for the first element, usually $n = 1$.
        \item\textbf{General case.} Now we suppose $P(n)$ to be true for any $n$, this is known as the 
            \textit{induction hypothesis}, so we need to prove that $P(n + 1)$ holds (\textit{induction thesis}).
    \end{itemize}
\end{defn}

\noindent Formally, the mathematical induction method is defined as follows.

\begin{prop}[Mathematical induction principle]\label{prop:math-induction}
    Let $X\subseteq\N$ be a subset of natural numbers such that $1\in X$ and $n\in X\implies n+1\in X$. Then, $X = \N$. 
\end{prop}

\begin{example}
    Let's prove the sum of the first $n$ natural numbers. In other words, prove that the proposition
    \begin{equation}\label{eq:induction-ex-hypothesis}
        P(n)\equiv\sum_{k=0}^n k = \frac{n(n + 1)}{2}
    \end{equation}
    holds for any $n\in\N$. We begin by checking that $P(1)$ is true, which is pretty trivial. Now suppose that $P(n)$ is
    true for some $n\in\N$, then we need to prove that $P(n + 1)$ holds, in other words, that
    \begin{equation}\label{eq:induction-ex-thesis}
        P(n + 1)\equiv\sum_{k = 0}^{n + 1} k = \frac{(n + 1)(n + 2)}{2}
    \end{equation}
    is true. To prove the induction thesis we begin from the left hand side of \eqref{eq:induction-ex-thesis} and we try 
    to reach the right hand side using our induction hypothesis \eqref{eq:induction-ex-hypothesis},
    \begin{align}
        \sum_{k=0}^{n+1} k &= \sum_{k=0}^n k + (n + 1) \overset{\eqref{eq:induction-ex-hypothesis}}{=} 
        \frac{n(n + 1)}{2} + (n + 1) \\  &= \frac{n(n + 1) + 2(n + 1)}{2} = \frac{(n + 1)(n + 2)}{2}.
    \end{align}
    Then, since $P(n + 1)$ holds, by the \nref{prop:math-induction}, the proposition $P(n)$ is true for all $n\in\N$.
\end{example}

