\topic{Algebraic structures}{Algebraic structures}
\subsection{The arithmetic of the integers}
There are two main operations between integer numbers. These are addition and product.

Now, we look at the integers paying attention to the addition operation, $\left( \Z, + \right) $.
\begin{remark}
    Addition is a binary operation: for all $a, b\in\Z\rightarrow\ a + b\in\Z$. We operate two by two elements.
\end{remark}

Because $a + b\in\Z$ we say that this binary operation is \textbf{closed in \Z}.

\begin{proposition}
    \textbf{(Associative property).} $\left( a + b \right) + c = a + \left( b + c \right) ,\ \forall a, b, c\in\Z$
\end{proposition}
\begin{proposition}
    \textbf{(Existence of an indentity element).} There's an integer $\varphi\in\Z \tq a + \varphi = \varphi + a = a,\ \forall a\in\Z$, and that integer is $\varphi = 0$.
\end{proposition}
\begin{proposition}
    \textbf{(Inverse element).} Every element $a\in\Z$ has an additive inverse, meaning that for every $a\in\Z$ there's $a'\in\Z\tq a + a' = a' + a = 0$, and in this case $a' = -a$.
\end{proposition}
\begin{proposition}
    \textbf{(Abelian or conmutative property).} $a + b = b + a,\ \forall a, b\in\Z$
\end{proposition}

Because the addition satisfies the first three properties we say that $\left( \Z, + \right) $ form a \textbf{group}. An because $\left( \Z, + \right) $ also satisfies the commutative property we say that $\left( \Z, + \right) $ form an \textbf{abelian group}.

\begin{definition}
    \textbf{(Group).} In general, if we have a set $X$ with some binary close operation $*$, satisfying the first three properties we say that $\left( X, * \right) $ is a \textbf{group}. In addition, if $\left( X, * \right) $ also satisfies the commutative property we say that $\left( X, * \right) $ is an \textbf{abelian group}.
\end{definition}

\begin{example}
    Let $\R^* = \R\setminus \{0\} $, we have that $\left( \R^*, \cdot \right) $ is an abelian group with identity $1$, multiplicative inverse $r^{-1} = \frac{1}{r}, r\in\R^*$.

    Another example of an abelian group is $\left( \R, + \right) $.
\end{example}

The product on $\Z$ is another binary closed operation as $a\cdot b\in\Z,\ \forall a, b\in\Z$.

\begin{proposition}
    \textbf{(Associative property).} $\left( a \cdot b \right) \cdot c = a \cdot \left( b \cdot c \right) ,\ \forall a, b, c\in\Z$
\end{proposition}

What do we do if we find the two operations in the same line?

\begin{proposition}
    \textbf{(Distributive property).} $a\cdot\left( b + c \right) = ab + ac$ and $\left( b + c \right)\cdot a = ba + ca,\ \forall a, b, c\in\Z $.
\end{proposition}

Because $\left( \Z, +, \cdot \right) $ satisfies the four properties for the addition and the associative for the product as well as the distibutive we say that $\left( \Z, +, \cdot \right) $ is a \textbf{ring}.

\begin{example}
    $\left( \N, +, \cdot \right) $ is not a ring. $\left( \N, + \right) $ is a binary closed operation that satisfies the associative property, has $0$ as an indentity element, but it doesn't have an additive inverse for each $n\in\N$. Therefore, $\left( \N, + \right) $ does not form a group.
\end{example}

Suppose that we have a set $X$ with two binary closed operations, $*$ and $\circ$. $\left(X, *, \circ  \right) $ is a ring if $\left( X, * \right) $ is an abelian group and if $\left( X, \circ \right) $ satisfies the associative property and $\left( X, *, \circ \right) $ satisfies the distributive property.

\subsection{Elemental algebraic structures}
\subsection{Greatest common divisor, Minimum common multiple, Euclid algorithm}
\subsection{Prime numbers. Factorization theorem}
\subsection{Congruences, divisibility, little Fermat's theorem}
