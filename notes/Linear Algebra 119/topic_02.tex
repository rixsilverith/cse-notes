\topic{Equivalence relations}{Equivalence relations}
%In mathematics sometimes there are situations in which it is convenient to establish relations among the elements of a set.
\hide{
\begin{example}[$\N= \{0, 1, 2\ldots\} $]
    \begin{equation}
        x\rela y (x \textrm{ is related to } y) \textrm{ if } x\le y
    \end{equation}
    Then, 2\rela 4 but not 4\rela 2.
\end{example}

Another example: Let $A = \{f:\R\to\R\} $. If the function $f$ is differentiable, then $f\rela g$ if $g = f'$.
}

\begin{definition}
    \textbf{(Relation).} A \textbf{relation} on a set $A$ is a non-empty subset $\rela\subset A\times A$ of ordered pairs $\left( x, y \right) $ where $x, y\in A$. 
\end{definition}

So technically any subset of $A\times A$ is a relation on $A$, however it should posses certain characteristics in order to be kind of interesting. Relations are generally used to compare two elements in some way. That is, we use them to determine whether two elements are \textit{related} in the manner specified.

\begin{example}[]
   Let the relation $x\rela y$ if $x\tq y$ with $x, y\in\N$. Then, that relation is defined by the following set
   \begin{equation}
       \rela = \{n, m : n\tq m\} 
   \end{equation}
\end{example}
\begin{remark}
    For any set $A$, both $\O$ and $A\times A$ are relations on $A$. The $\O$ relation doesn't relate elements to anything, not even themselves. On the other hand, the relation $A\times A$ relates every element to every element of $A$.
\end{remark}

\begin{definition}
    \textbf{(Equivalence relation).} A relation $\sim$ on a non-empty set $A$ is an \textbf{equivalence relation} if all of the following conditions are satisfied.
    \begin{itemize}
        \item\textbf{Reflexive.} If $x\sim x\, \forall x\in A$.
        \item\textbf{Symmetric.} If $x\sim y$, then $y\sim x$.
        \item\textbf{Transitive.} If $x\sim y$ and $y\sim z$, then $x\sim z$.
    \end{itemize}
\end{definition}

\begin{example}
    Let $x\sim y$ be a relation between $x$ and $y$ if $x\leq y$ with $x, y\in\N$. In order to check if this is an equivalence relation we should check if it satisfies the reflexive, symmetric and transitive conditions.

    \begin{itemize}
        \item[(a)] $x\sim x \iff x\leq x,\ \forall x\in\N$, then the relation $\sim$ is reflexive.
        \item[(b)] If $x\sim y$ then $y\sim x$? No. Just take $2\sim 4 \iff 2\leqslant 4$ but $4\sim 2 \iff 4\nleqslant 2$. Then, the relation is not symmetric, and therefore it is not an equivalence relation.
        \item[(c)] If $x\sim y$ and $y\sim z \implies x\sim z$?
            \begin{equation}
                \begin{rcases}
                    x\sim y \quad\implies\quad x\leqslant y \\
                    y\sim z \quad\implies\quad y\leqslant z
                \end{rcases} \quad\implies\quad x\leqslant z \iff x\sim z\quad\implies\quad\sim\textrm{ is transitive.}
            \end{equation}
    \end{itemize}
    Because the reflexive and the transitive conditions are met, but not the symmetric, the $\sim$ relation is not an equivalence relation.
    \qed
\end{example}

Sometimes when working with elements in a set, it is convenient to consider that some of them are \textit{equal}, even though they are not. To declare two elements as equal we define a relation in the set and, to make sure that we don't get something ilogical, we need this relation to be an equivalence relation.

\begin{definition}
\textbf{(Equivalence class).} Given an equivalence relation $\sim$ on a set $A$ and an element $a\in A$, the \textbf{equivalence class} of $a$ is the set
    \begin{equation}
        [a] = \overline{a} := \{x\in A \tq a\sim x\} 
    \end{equation}
\end{definition}

In other words, we can say that the equivalence class of the element $a$ of a set $A$ is the set of all elements$x$ in the set $A$ which are related to the element $a$.

\begin{definition}
    \textbf{(Quotient set).} The quotient set of a set $A$ by the equivalence relation $\sim$, denoted by $A / \sim$, is the set of all the equivalence classes.
\end{definition}
\begin{example}
    Let $x\sim y$ be an equivalence relation between $x$ and $y$ if $3\tq \left( x - y \right)$. Now, we can think about a new set where
    \begin{align}
        0&\equiv 3\equiv -3\equiv 6\equiv 9\equiv\ldots\quad\rightarrow\quad\overline{0} \\
        1&\equiv 4\equiv 7\equiv 13\equiv\ldots\quad\rightarrow\quad\overline{1} \\
        2&\equiv 5\equiv 8\equiv\ldots\quad\rightarrow\quad\overline{2}
    \end{align}
    \begin{equation}
        \Z / \sim = \{[0], [1], [2]\} 
    \end{equation}
    %This is known as  \textbf{quotient set}.
\end{example}
\begin{example}
    Let $x\rela y$ be a relation between $x$ and $y$ if $\abs{x} = \abs{y}$ where $x,y\in\R$. Is this an equivalence relation?
    \begin{itemize}
        \item[(a)] $x\rela x\iff \abs{x} = \abs{x}$. Then, the relation $x\rela y$ is reflexive $\forall x\in\R$.
        \item[(b)] $x\rela y\implies \abs{x} = \abs{y}\implies y\rela x$. Then, the relation $x\rela y$ is symmetric because $\abs{y} = \abs{x}$.
        \item[(c)] If $x\rela y$ and $y\rela z\implies x\rela z$?
            \begin{equation}
                \begin{rcases}
                    \textrm{Since } x\rela y\quad\implies\quad \abs{x} = \abs{y} \\
                    \textrm{Since } y\rela z\quad\implies\quad \abs{y} = \abs{z}
                \end{rcases}\quad\implies\quad \abs{x} = \abs{z}\quad\implies\quad x\rela z
            \end{equation}
            Then, the relation $x\rela y$ is transitive. 
    \end{itemize}
Therefore, it is an equivalence relation.
\begin{align}
    \overline{1} &= \{r\in\R\tq 1\rela r\} = \{1, -1\} \\
    \overline{0} &= \{0\} \\
    \overline{-\sqrt{2}} &= \{\sqrt{2}, -\sqrt{2}\} \\
    \vdots \\
    \R / \R &= \{\overline{r}\tq r\in\R\} = \R_{\geq 0} 
\end{align}
\begin{itemize}
    \item Each $r\in\R$ is in the same equivalence class as some element in $\R_{\geq 0}$.
    \item Two different reals in $\R_{\geq 0}$ correspond to two different classes.
\end{itemize}
\end{example}
%\subsection{Equivalence relations as set partitions}

